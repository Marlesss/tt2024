\documentclass[aspectratio=169]{beamer}
\setbeamertemplate{navigation symbols}{}
\usepackage{graphicx}
\usepackage{stmaryrd}
\usepackage[T1,T2A]{fontenc}
\usepackage[utf8]{inputenc}
\usepackage[english,russian]{babel}
\usepackage{amsmath}
\usepackage{amsfonts}
\usepackage{amssymb}
\usepackage{makeidx}
\usepackage{verbatim}
\usepackage{amsthm}
\usepackage{bnf}
\usepackage{tikz}
\usepackage{enumerate}
\usepackage{mathtext}
\usepackage{mathtools}
\usepackage{mathabx}
%\usepackage[left=2cm,right=2cm,top=2cm,bottom=2cm,bindingoffset=0cm]{geometry}
\usepackage{proof}
%\usepackage{paracol}
%\usepackage{enumitem}
\usepackage{color}
\usepackage{colortbl}
\usepackage{minted}
%\usepackage{hyperref}
\usetikzlibrary{graphs}
\usetikzlibrary{graphs.standard}
\usetikzlibrary{automata,positioning}
\usepackage{float}

\begin{document}

%\theoremstyle{dfn}
\newtheorem{dfn}{Определение}[section]
\newtheorem{nte}{Замечание}[section]

\newtheorem{axiom}{Аксиома}[section]
\newtheorem{thm}{Теорема}[section]
\newtheorem{lmm}[theorem]{Лемма}
\newtheorem{statement}{Утверждение}[section]
\newtheorem{oun_paragraph}{Пункт}[section]
\newtheorem{cons}{Следствие}[section]
\newtheorem*{exm}{Пример}

\newcommand{\comb}[1]{\operatorname{\bf{\textrm{#1}}}}
\newcommand{\func}[1]{\operatorname{#1}}
\newcommand{\reduction}[1]{{\color{OrangeRed}#1}}
\newcommand{\set}[1]{\left\{#1\right\}}

\def\from#1{\par \parbox{0.7\textwidth}{\par \hfill\raggedleft \it #1}} 

\begin{frame}{}
\begin{center}\Large Лекция 3.\\ \Large Нормализуемость $\lambda_\rightarrow$.\\Система F\end{center}
\end{frame}

\begin{frame}{Нормализуемость}
\begin{dfn}
\begin{itemize}
\item Терм $A$ назовём слабо нормализуемым, если существует последовательность редукций,
приводящих его в нормальную форму.
\item Терм $A$ назовём сильно нормализуемым, если не существует бесконечной последовательности
его редукций.
\item Исчисление назовём сильно нормализуемым, если любой его терм сильно нормализуем.
\end{itemize}
\end{dfn}

\vspace{-0.3cm}
\begin{thm}Бестиповое лямбда-исчисление не является сильно нормализуемым\end{thm}
\vspace{-0.3cm}
\begin{proof}$\Omega \rightarrow_\beta \Omega$\end{proof}

\begin{thm}Просто типизированное лямбда-исчисление является сильно нормализуемым\end{thm}
\end{frame}

\begin{comment}
\begin{frame}{$SN_\sigma$: сильно нормализуемые термы типа $\sigma$}
\begin{dfn} Будем записывать $SN_\alpha(T)$, если \begin{enumerate}
\item $\vdash T : \alpha$
\item $T$ сильно нормализуем.
\end{enumerate}\vspace{0.5cm}

$SN_{\sigma\rightarrow\tau}(T)$, если \begin{enumerate}
\item $\vdash T : \sigma\rightarrow\tau$
\item $T$ сильно нормализуем
\item Если $SN_\sigma(P)$, то $SN_\tau(T\ P)$.
Неформальное пояснение: функция гарантированно заканчивает работу на любых сильно нормализуемых аргументах.
\end{enumerate}

\end{dfn}
\end{frame}

\begin{frame}[fragile]{Пример 1}

{\color{gray}$SN_\zeta(T)$, если:
$\vdash T:\zeta$, $T$ сильно нормализуем, при $\zeta=\sigma\rightarrow\tau$ и $SN_\sigma(P)$
выполнено $SN_\tau(T\ P)$
}

\begin{exm}
$SN_{\alpha\rightarrow\alpha}(\lambda x.x)$:
\begin{enumerate}
\item $\vdash \lambda x.x : \alpha\rightarrow\alpha$
\item $\lambda x.x$ находится в нормальной форме
\item Каждой цепочке редукций $I\ P$ соответствует цепочка редукций $P$ длины $k-1$ или $k$
(редукции внутри $P$ оставляем, редукцию $I\ Q \rightarrow_\beta Q$ пропускаем).
Отсюда, если $SN_\alpha(P)$, то $I\ P$ сильно нормализуемо и $\vdash I\ P : \alpha$.
\end{enumerate}
\end{exm}
\end{frame}

\begin{frame}[fragile]{Пример 2}

\begin{minipage}{10cm}{\color{gray}$SN_\zeta(T)$, если:\\
 (1) $\vdash T:\zeta$, (2) $T$ сильно нормализуем, \\(3) при $\zeta=\sigma\rightarrow\tau$ и $SN_\sigma(P)$
выполнено $SN_\tau(T\ P)$}
\end{minipage}

\begin{exm}
Важность третьего условия: не обязательно результат подстановки сильно нормализуемого терма 
в сильно нормализуемый терм сильно нормализуем.

\begin{itemize}
\item $\lambda x.x\ x$ сильно нормализуем.
\item Бывает так: $(\lambda x.x\ x)\ I \rightarrow_\beta I\ I \rightarrow_\beta I$
\item Но бывает и так: $(\lambda x.x\ x)\ (\lambda x.x\ x)$.
\end{itemize}
\end{exm}

\end{frame}

\begin{frame}
\begin{thm}Если $\vdash S:\sigma$ и $S\rightarrow_\beta T$, то $SN_\sigma(S)$ тогда и только тогда, когда $SN_\sigma(T)$.\end{thm}
\begin{proof}Лемма о редукции типа, индукция по структуре $\sigma$

$(\lambda x.P) Q -> P [x:=Q]$
Если $\sigma = \alpha$, то $SN_\alpha(T)$ означает, что 

\end{proof}
\end{frame}

\begin{frame}{Применение типизированного терма к $SN$-аргументам само $SN$}
Если $\{x_i : \sigma_i\} \vdash Q : \tau$ и $SN_{\sigma_i}(P_i)$, то $SN_\tau(Q[x_1 := P_1,\dots,x_n := P_n])$.
Индукция по структуре доказательства $x_i : \sigma_i \vdash Q:\tau$.
\begin{enumerate}
\item $$\infer{\vdash x_k : \sigma_k}{\{x_i : \sigma_i\}}$$
\item $$\infer{\Gamma\vdash M\ N : \psi}{\Gamma\vdash M : \varphi\rightarrow\psi\quad\quad N : \psi}$$
По индукции, $SN_{\varphi\rightarrow\psi}(M[{x_i := P_i}])$. Также, по определению имеем $SN_{\varphi\rightarrow\psi}$.
Отсюда $SN_\psi(M[{x_i := P_i}]\ N[{x_i := P_i}])$
\end{enumerate}
\end{frame}

\begin{frame}{Типизированный терм является сильно нормализованым}
\begin{thm}Если $\vdash T:\tau$, то $T сильно нормализуем.$\end{thm}
\begin{proof}По вырожденному случаю леммы о применении типизированного терма $SN_\tau(T[])$ (при пустом списке замен).
Тогда по определению $SN_\tau(T)$ имеем сильную нормализуемость $T$.\end{proof}
\end{frame}
\end{comment}

\begin{frame}{Сильно нормализуемые множества}
\begin{dfn}SN --- множество всех сильно нормализуемых лямбда-термов.

Насыщенное множество $\mathcal{X} \subseteq SN$ --- такое, что: \begin{enumerate}
\item для любых $n \ge 0$ и $M_1, \dots, M_n \in SN$ $$x\ M_1 \dots M_n \in \mathcal{X}$$
\item для любых $n \ge 1$, $M_1, \dots, M_n \in SN$ и $N \in \Lambda$ $$N[x := M_1]\ M_2 \dots M_n \in \mathcal{X}\text{ влечёт }(\lambda x.N)\ M_1\ M_2\dots M_n \in \mathcal{X}$$
\end{enumerate}
\end{dfn}

\vspace{-0.7cm}
\begin{lmm}
$SN$ --- насыщенное. 
\end{lmm}
Интересен пункт 2: если $N[x := M_1]\ M_2\dots M_n \in SN$, то $\underline{(\lambda x.N)\ M_1}\ M_2\dots M_n \in SN$. Подстановка подчёркнутого 
возвращает к редукции посылки, бесконечная <<локальная>> подстановка может быть повторена с посылкой.
\end{frame}

\begin{frame}{}
\begin{dfn}
Если $\mathcal{A},\mathcal{B} \subseteq \Lambda$, то $\mathcal{A} \rightarrow \mathcal{B} = \{ X \in \Lambda\ |\ \forall Y \in \mathcal{A}\ .\ X\ Y \in \mathcal{B}\}$
\end{dfn}

\begin{exm}
$\{ \lambda x.\lambda y.x \} \rightarrow \{ X\ |\ X =_\beta \lambda x.\lambda y.y \} = \{ Not, \lambda t.F, Xor\ T, \dots \}$
\end{exm}

\begin{dfn}
$$\llbracket \sigma \rrbracket = \left\{\begin{array}{ll}SN, & \sigma = \alpha\\
  \llbracket \tau_1 \rrbracket \rightarrow \llbracket \tau_2 \rrbracket, & \sigma = \tau_1 \rightarrow \tau_2\end{array}\right.$$
\end{dfn}

\begin{lmm}
Если $\mathcal{A}, \mathcal{B}$ насыщены, то $\mathcal{A}\rightarrow\mathcal{B}$ насыщено\\
$\llbracket\sigma\rrbracket$ насыщено.
\end{lmm}

\begin{lmm}$\llbracket\sigma\rrbracket\subseteq SN$\end{lmm}

\end{frame}

\begin{frame}{Оценка}
\begin{dfn}
Оценка $\rho: \mathcal{V} \rightarrow \Lambda$ --- отображение переменных в лямбда-термы.

$M_\rho := M[x_1 := \rho(x_1), \dots, x_n := \rho(x_n)]$, где $x_i$ --- все свободные переменные $M$.

Будем писать $\rho \models M:\sigma$, если $M_\rho \in \llbracket\sigma\rrbracket$.
Будем писать $\rho \models \Gamma$, если $\rho(x) \in \llbracket\sigma\rrbracket$ для всех $x : \sigma \in \Gamma$.

$\Gamma \models M:\sigma$, если для любой оценки $\rho$ из $\rho\models\Gamma$ следует $\rho \models M:\sigma$.
\end{dfn}

\begin{thm}$\Gamma\vdash M:\sigma$ влечёт $\Gamma\models M:\sigma$.
\end{thm}

Доказательство индукцией по структуре вывода $\Gamma\vdash M:\sigma$ со следующим разбором случаев.

\end{frame}

\begin{frame}{Аксиома}
Вывод имеет вид:

$$\infer{\Gamma, x : \sigma \vdash x:\sigma}{}$$

Фиксируем $\rho \models \Gamma\cup\{x : \sigma\}$, тогда
$x_\rho = \rho(x)\in\llbracket\sigma\rrbracket$

Отсюда $\Gamma, x : \sigma \models x : \sigma$
\end{frame}

\begin{frame}{Применение}
Вывод имеет вид:
$$\infer{\Gamma\vdash M\ N:\tau}{\Gamma\vdash M:\sigma\rightarrow\tau\quad\quad\Gamma\vdash N:\sigma}$$
Фиксируем $\rho \models \Gamma$. По индукционному предположению, $\Gamma\models M:\sigma\rightarrow\tau$ и
$\Gamma\models N:\sigma$, так что $\rho\models M:\sigma\rightarrow\tau$ и $\rho\models N:\sigma$, что означает, что
$M_\rho \in \llbracket\sigma\rrbracket\rightarrow\llbracket\tau\rrbracket$ и $N_\rho \in \llbracket\sigma\rrbracket$.
Тогда $(M\ N)_\rho = M_\rho\ N_\rho \in \llbracket\tau\rrbracket$.
\end{frame}
\begin{frame}{Абстракция}
Вывод имеет вид:
$$\infer[x \notin FV(\Gamma)]{\Gamma\vdash\lambda x.M : \sigma\rightarrow\tau}{\Gamma,x:\sigma\vdash M:\tau}$$

Пусть $\rho\models\Gamma$.
Чтобы показать $(\lambda x.M)_\rho \in \llbracket\sigma\rightarrow\tau\rrbracket$, надо для всех $N \in \llbracket\sigma\rrbracket$
показать $(\lambda x.M)_\rho\ N \in \llbracket\tau\rrbracket$.

Фиксируем $N \in \llbracket\sigma\rrbracket$. Тогда $\rho^{x := N}\models\Gamma,x:\sigma$. 
По индукционному предположению, $\Gamma,x:\sigma\models M:\tau$, так что $\rho^{x := N}\models M:\tau$ (по определению $\models$).
То есть, $M_{\rho^{x := N}} \in \llbracket\tau\rrbracket$. Произведём редукцию:
$$(\lambda x.M)_\rho N = (\lambda x.M)^{y_1 := \rho(y_1), \dots, y_n := \rho(y_n)}\ N \rightarrow_\beta M^{y_1 := \rho(y_1), \dots, y_n := \rho(y_n), x:=N}
 = M_{\rho^{x := N}}$$

Заметим, $N \in \llbracket \sigma \rrbracket \subseteq SN$ и $M_{\rho^{x := N}} \in \llbracket\tau\rrbracket$.
Заметим ещё, что $M_{\rho^{x:=N}} = M_\rho[x := N]$.
По определению насыщенного множества из $M_\rho[x := N] \in \llbracket\tau\rrbracket$ следует требуемое $(\lambda x.M)_\rho\ N \in \llbracket\tau\rrbracket$.
\end{frame}
\begin{frame}{Основная теорема}
\begin{thm}$\Gamma\vdash M:\sigma$ влечёт $M \in SN$\end{thm}
\begin{proof}
По предыдущей теореме, $\Gamma\models M:\sigma$. Построим <<тождественную>> оценку, $\rho(x) = x$ для всех $x:\tau \in \Gamma$.

Рассмотрим каждый $x : \tau$ из контекста.
По лемме выше, $\llbracket\tau\rrbracket$ насыщенное. По определению насыщенного, $x \in \llbracket\tau\rrbracket$.
Поэтому $\rho\models\Gamma$.

Поскольку $\Gamma\models M:\sigma$, то $M = M_\rho \in \llbracket \sigma \rrbracket$. А по лемме выше, $\llbracket \sigma \rrbracket \subseteq SN$.
\end{proof}
\end{frame}

\begin{frame}{О свойстве сильной нормализуемости}

Правило сечения в $S_\infty$ (без одной боковой формулы):
$$\infer{\sigma}{\sigma\vee\neg\beta\quad\quad\beta}$$
Или перепишем в привычной грамматике (подобно Modus Ponens):
$$\infer{\sigma}{\beta\rightarrow\sigma\quad\quad\beta}$$
И заметим нечто похожее в просто-типизированном лямбда-исчислении:
$$\infer[\beta-\text{редекс}]{(\lambda x.P)\ Q:\sigma}{(\lambda x.P) : \tau\rightarrow\sigma\quad\quad Q:\tau}$$

Поэтому добавим пункты к изоморфизму Карри-Ховарда:
\begin{center}\begin{tabular}{l|l}
Логика & $\lambda_\rightarrow$ \\\hline
Правило сечения, M.P. & Бета-редекс\\
Устранение сечения & Бета-редукция\\
Теорема об устранении сечений & Нормализуемость
\end{tabular}\end{center}

\end{frame}

\begin{frame}{ИИП второго порядка}
\begin{itemize}
\item Алфавит: $a-z$, $\vee$, $\&$, $\rightarrow$, $\neg$, $\forall$, $\exists$.
\item Метапеременные: $\alpha$ для формул, $p,x,y,z$ для переменных.
\item $F ::= p\ |\ (F\star F)\ |\ (\forall p.F)\ |\ (\exists p.F)\ |\ \bot$
\item Сокращения записи: приоритеты как в ИИВ, подкванторное выражение продолжается направо настолько, насколько возможно.
\end{itemize}

\begin{exm}
$\forall p.\forall q.p \rightarrow q \rightarrow p$
\end{exm}
\end{frame}

\begin{frame}{Теория доказательств}
    Правила вывода совпадают с правилами для ИИВ, добавлены 4 новых:

    \[ \dfrac{\Gamma\vdash\varphi}{\Gamma\vdash\forall p.\varphi} (p\notin FV(\Gamma)) \qquad
        \dfrac{\Gamma\vdash\forall p.\varphi}{\Gamma\vdash\varphi[p:=\theta]} \]


    \[ \dfrac{\Gamma\vdash\varphi[p := \theta]}{\Gamma\vdash\exists p.\varphi} \qquad
       \dfrac{\Gamma\vdash\exists p.\varphi\quad\Gamma,\varphi\vdash\psi}{\Gamma\vdash\psi} (p \notin FV(\Gamma,\psi)) \]
\end{frame}

\begin{frame}{Теория моделей}
Простая неполная модель.

$$V = \{ \text{И}, \text{Л} \}$$

%\begin{center}
%	 	$\llbracket p\rrbracket=p$, т. е. $\llbracket p\rrbracket^{p = x} = x$ \\
%\end{center}
 	
 	
 \begin{center}
 		\begin{equation*}
 		\llbracket P\rightarrow Q\rrbracket = 
 		\begin{cases}
 			\text{Л}, \llbracket P\rrbracket = \text{И}, \llbracket Q\rrbracket = \text{Л} \\
 			\text{И}, \text{иначе}
 		\end{cases}
 	\end{equation*}
 \end{center}
 	
 	
 	\begin{equation*}
 		\llbracket\forall p.Q\rrbracket = 
 		\begin{cases}
 			\text{И}, \llbracket Q\rrbracket^{p:=\text{Л, И}} = \text{И} \\
 			\text{Л}, \text{иначе}
 		\end{cases}
 	\end{equation*}
% 	Эта модель корректна, но не полна.


\end{frame}

\begin{frame}[fragile]{Выразимость всех связок через $\forall$, $\rightarrow$}
Заметим, что достаточно определить связки $\forall$ и $\rightarrow$.\vspace{0.5cm}

\begin{tabular}{ll}
 Связка & Способ выразить \\
 $\alpha\&\beta$ & $\forall p.(\alpha\rightarrow\beta\rightarrow p)\rightarrow p$\\
 $\alpha\vee\beta$ & $\forall p.(\alpha\rightarrow p)\rightarrow (\beta\rightarrow p) \rightarrow p$\\
 $\bot$ & $\forall p.p$ \\
 $\exists p.\varphi$ & $\forall f.(\forall p.\varphi\rightarrow f) \rightarrow f$
\end{tabular}

\vspace{0.5cm}
С так определёнными связками оказывается возможно показать все правила вывода.
Например, примем $\alpha\&\beta$ за $\forall p.(\alpha\rightarrow\beta\rightarrow p)\rightarrow p$ и покажем,
что из $\alpha\&\beta$ следует $\alpha$:
$$\infer{\alpha}{
    \infer{\vdash\alpha\rightarrow\beta\rightarrow\alpha}{\infer{\alpha\vdash\beta\rightarrow\alpha}{\infer{\alpha,\beta\vdash\alpha}{}}}\quad\quad
    \infer[p := \alpha]{\vdash(\alpha\rightarrow\beta\rightarrow \alpha)\rightarrow \alpha}{\vdash\forall p.(\alpha\rightarrow\beta\rightarrow p)\rightarrow p}
}$$

\end{frame}

\begin{frame}[fragile]{Система F}
\begin{dfn}
        Типы в системе F:
 	\begin{equation*}
 	\tau =
 	\begin{cases}
 	\alpha,\beta,\gamma ...\quad(\text{атомарные типы}) \\
 	\tau\rightarrow\tau \\
 	\forall\alpha.\tau\qquad(\alpha\text{ - переменная})
 	\end{cases}
 	\end{equation*}
 \end{dfn}

 \begin{dfn}
        Пред-лямбда-терм в системе F (типизировано по Чёрчу):
 	\begin{center}
 		$F ::= x\ |\ (\lambda x^{\tau}.F)\ |\ (F\ F)\ |\ (\Lambda\alpha.F)\ |\ (F\ \tau)$ 
 	\end{center}
 \end{dfn}
\end{frame}

\begin{frame}[fragile]{Типовая абстракция и применение}
   Примеры соответствующих конструкций из C++.
   \begin{itemize}
    \item Типовая абстракция, $\Lambda \tau.W$:
    \begin{verbatim}
template<typename t>
class W {
    t x;
}
    \end{verbatim}
 	
    \item Типовое применение, $W\ int$:
    \begin{verbatim}
W<int> w_test;
    \end{verbatim}
  \end{itemize}
\end{frame}
 	
\begin{frame}
 	В системе F определены следующие правила вывода: \\ \\
  	\noindent
 	{$\dfrac{}{\Gamma,x:\tau\vdash x:\tau}\qquad\qquad$} 
 	{$\dfrac{\Gamma\vdash M:\sigma\rightarrow\tau\qquad\Gamma\vdash N:\sigma}{\Gamma\vdash M N:\tau}$}\\  \\ \\
 	{$\dfrac{\Gamma,x:\tau\vdash M:\sigma}{\Gamma\vdash\lambda x^{\tau}.M:\tau\rightarrow\sigma}\quad(x\notin FV(\Gamma))$}\\ \\ \\
 	{$\dfrac{\Gamma\vdash M:\sigma}{\Gamma\vdash\Lambda\alpha.M:\forall\alpha.\sigma}\quad(\alpha\notin FV(\Gamma))\qquad$}
 	 $\dfrac{\Gamma\vdash M:\forall\alpha.\sigma}{\Gamma\vdash M\tau:\sigma[\alpha:=\tau]}$
 	\\
 	
    \emph{Начнем с $\beta$-редукции:}
    \begin{enumerate}
        \item Типовая $\beta$-редукция: $(\Lambda\alpha.M^{\sigma})\tau\rightarrow_\beta M[\alpha:= \tau]:\sigma[\alpha:= \tau]$
        \item Классическая $\beta$-редукция: $(\lambda x^{\sigma}.M)^{\sigma\rightarrow\tau}X\rightarrow_\beta M[x:=X]:\tau$ 
    \end{enumerate}
 	
\end{frame}

\end{document}
