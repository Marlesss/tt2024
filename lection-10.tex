\documentclass[aspectratio=169,dvipsnames,usenames]{beamer}
\usepackage{graphicx}
\usepackage{stmaryrd}
\usepackage[T1,T2A]{fontenc}
\usepackage[utf8]{inputenc}
\usepackage[english,russian]{babel}
\usepackage{amsmath}
\usepackage{amsfonts}
\usepackage{amssymb}
\usepackage{makeidx}
\usepackage{verbatim}
\usepackage{amsthm}
%\usepackage{bnf}
\usepackage{tikz}
\usepackage{enumerate}
\usepackage{mathtext}
\usepackage{mathtools}
\usepackage{mathabx}
%\usepackage[left=2cm,right=2cm,top=2cm,bottom=2cm,bindingoffset=0cm]{geometry}
\usepackage{proof}
%\usepackage{paracol}
%\usepackage{enumitem}
\usepackage{xcolor}
\usepackage{colortbl}
%\usepackage{minted}
%\usepackage{hyperref}
\setbeamertemplate{navigation symbols}{}
\usetikzlibrary{graphs}
\usetikzlibrary{graphs.standard}
\usetikzlibrary{automata,positioning}
\usepackage{float}

\begin{document}

%\theoremstyle{dfn}
\newtheorem{dfn}{Определение}[section]
\newtheorem{nte}{Замечание}[section]

\newtheorem{axiom}{Аксиома}[section]
\newtheorem{thm}{Теорема}[section]
\newtheorem{lmm}[theorem]{Лемма}
\newtheorem{statement}{Утверждение}[section]
\newtheorem{oun_paragraph}{Пункт}[section]
\newtheorem{cons}{Следствие}[section]
\newtheorem*{exm}{Пример}

\newcommand{\comb}[1]{\operatorname{\mathcal{#1}}}
\newcommand{\func}[1]{\operatorname{#1}}
\newcommand{\reduction}[1]{{\color{OrangeRed}#1}}
\newcommand{\set}[1]{\left\{#1\right\}}

\def\from#1{\par \parbox{0.7\textwidth}{\par \hfill\raggedleft \it #1}} 

\begin{frame}{}
\begin{center}
{\LARGE Парадокс Жирара}
\end{center}
\end{frame}

\begin{frame}{Расширение обобщённых типовых систем}

Расширение обобщённых типовых систем, 5 уровней:

\vspace{1cm}
\begin{tabular}{lll}
№ & название & обозначение, примеры\\\hline
0 & термы: доказательства & x, $\lambda p.q$, ...\\
1 & типы: утверждения & $1=1$ \\
2 & рода: семейства утверждений & $\star$ \\
3 & сорт <<квадратик>> & $\Box$ (единственный представитель) \\
4 & сорт <<треугольник>> & $\triangle$ (единственный представитель)
\end{tabular}

\end{frame}
\begin{frame}{Обобщённые типовые системы, системы $U$ и $U^-$}

\begin{tabular}{lll}
Название & Аксиомы & Правила\\\hline
$\lambda HOL$ & $\star : \Box$, $\Box : \triangle$ & $(\star,\star)$, $(\Box,\Box)$, $(\Box,\star)$ \\
             & $\star : \Box$, $\Box : \triangle$ & $(\star,\star)$, $(\Box,\Box)$, $(\Box,\star)$, $(\triangle,\star)$ \\\hline
$\lambda U^-$ & $\star : \Box$, $\Box : \triangle$ & $(\star,\star)$, $(\Box,\Box)$, $(\Box,\star)$, $(\triangle,\Box)$ \\
$\lambda U$ & $\star : \Box$, $\Box : \triangle$ & $(\star,\star)$, $(\Box,\Box)$, $(\Box,\star)$, $(\triangle,\Box)$, $(\triangle,\star)$ 
\end{tabular}

\end{frame}

\begin{frame}{Парадокс Бурали-Форте}

\begin{thm}Множества всех ординалов не существует\end{thm}
\begin{proof} %(схематичное изложение)
Пусть $\Omega$ --- множество всех ординалов. 
\begin{enumerate}
\item Легко заметить, что $\Omega$ является ординалом:
\begin{itemize}
\item Все ординалы сравнимы (если есть аксиома выбора) --- линейный порядок
\item Если множество ординалов $X \subseteq \Omega$ непусто $(x \in X)$, то в $\min X = \min (x' \cap X)$.
Поскольку $x'$ --- ординал, то в любом его непустом подмножестве есть минимум --- полный порядок.
\item Если $y \in x$ и $x \in \Omega$, то $y$ ординал --- транзитивность.
\end{itemize}
\item Однако, любой ординал $\alpha \in \Omega$, то есть $\Omega < \Omega$
\item Однако, это есть противоречие с определением строгого порядка.
\end{enumerate}
\end{proof}
\end{frame}

\begin{frame}{Порядковые типы}

\begin{definition}Множества $(A,\prec)$ и $(B,\sqsubset)$ имеют одинаковый \emph{порядковый тип}, если существует биекция между ними, сохраняющая порядок:
$$\forall a\in A.\forall b \in A.a \prec b \leftrightarrow f(a) \sqsubset f(b)$$
\end{definition}

\begin{definition}Операции перехода от множества к порядковому типу и назад:
\begin{enumerate}
\item $\sigma$ --- по порядковому типу (по соответствующему ординалу) получить вполне упорядоченное множество: $\sigma\alpha = \{ t: \text{ординал} | t < \alpha \}$
\item $\tau$ --- по вполне упорядоченному множеству получить его порядковый тип: $\tau X$
\end{enumerate}

Заметим, что (здесь $X$ --- некоторое множество ординалов):
 $$\sigma\tau X = \{ \beta : \text{ординал} | \beta < \tau X\} = \{ \tau\sigma\alpha | \alpha \in X \}$$
\end{definition}

\end{frame}
\begin{frame}{Парадоксальные универсумы}

Напомним, что $X \rightarrow Y := \Pi x^X.Y$.

\begin{definition}\emph{Булеан} множества $S$ (power set) --- множество всех предикатов на $S$:
$$\wp: S \rightarrow \star$$
\end{definition}


Рассмотрим $\sigma: \mathcal{U}\rightarrow\wp\mathcal{U}$ и $\tau: \wp\mathcal{U}\rightarrow\mathcal{U}$

\begin{definition}
$\mathcal{U}$ парадоксален, если для всех $X \in \wp\mathcal{U}$ выполнено $$\sigma\tau X = \{\tau\sigma x | x \in X \}$$
\end{definition}

%\vspace{3cm}

\end{frame}
\begin{frame}{Предшественники}

Фиксируем парадоксальный универсум $\mathcal{U}$ вместе с отображениями $\sigma$ и $\tau$.

\begin{definition}
Назовём $y$ \emph{предшественником} $x$ ($y < x$), если $y \in \sigma x$.
\end{definition}

\begin{lemma}[о выражении предшественника]
Если $t < \tau\sigma x$, то $t = \tau\sigma y$ для некоторого $y < x$
\end{lemma}
\begin{proof}
Пусть $X = \sigma x = \{ y | y < x \}$; это множество всех предшественников $x$.

$t < \tau\sigma x$, то есть $t \in \sigma\tau\sigma x$, то есть $t \in \sigma\tau X$, то есть (по парадоксальности) $t \in \{\tau\sigma y | y \in X \}$,
то есть $t = \tau\sigma y$ для некоторого $y \in X$, то есть $t = \tau\sigma y$ при $y < X$.
%$\sigma\tau X = \sigma\{ \tau\sigma y | y < x \} = \sigma X$
%По определению предшествования, $t = \sigma\tau\sigma x$,
%Тогда $\sigma\tau X = \{\tau\sigma y | y < x \}$ (по парадоксальности),
%поэтому $t$ 
\end{proof}
\begin{lemma}
Если $p < q$, то $\tau\sigma p < \tau\sigma q$
\end{lemma}

\end{frame}

\begin{frame}{Индуктивность и фундированность}

\begin{definition} Рассмотрим универсум $\mathcal{U}$. Индуктивность множеств и фундированность:

\begin{enumerate}
\item $X \subseteq \mathcal{U}$ --- \emph{индуктивное}, если всякий $x$, предшественники которого содержатся в $X$, сам содержится в $X$:
$$\text{Если для всех }y < x\text{ выполнено }y \in X\text{, то }x \in X$$

\item $x \in \mathcal{U}$ --- \emph{фундированный}, если $x$ принадлежит всем индуктивным множествам.
\end{enumerate}
\end{definition}
\end{frame}

\begin{frame}{Множество $\Omega$}
\begin{definition} Фиксируем некоторый универсум $\langle\mathcal{U},\sigma,\tau\rangle$. Тогда $\Omega = \tau\{x | x \text{ фундирован}, x\in\mathcal{U}\}$ --- 
порядковый тип множества всех фундированных множеств.
\end{definition}

Легко заметить, что $\Omega\in\mathcal{U}$.

\begin{lmm}Если $X$ индуктивно, то и $T = \{y | \tau\sigma y \in X\}$ также индуктивно.\end{lmm}
\begin{proof}
Для индуктивности $T$ нужно показать, что если $y < x$ влечёт $y \in T$, то $x \in T$.
Фиксируем $x$, что $\tau\sigma y \in X$ для всех $y < x$.
Тогда заметим, что если $k < \tau\sigma x$, то найдётся такой $y$, что $k = \tau\sigma y$ и $y < x$
(лемма о выражении предшественника), то есть $k \in X$. Значит (индуктивность $X$), $\tau\sigma x \in X$, то есть $x \in T$.
\end{proof}


\end{frame}

\begin{frame}{$\Omega$ фундировано}
\begin{lmm} В парадоксальном универсуме $(\sigma\tau X = \{\tau\sigma x| x \in X\})$ множество $\Omega$ фундировано.
\end{lmm}
%Фундированность $\Omega$ означает, что любое индуктивное множество его содержит. 
\begin{proof}
\begin{itemize}
\item Фиксируем индуктивное $X \subseteq \mathcal{U}$. Чтобы показать $\Omega \in X$, по определению индуктивности надо показать, что если $t < \Omega$, то $t \in X$.
Пусть $F$ --- множество всех фундированных элементов, $F \subseteq \mathcal{U}$.

\item Пусть $t < \Omega$, тогда $t \in \sigma\Omega$ (определение предшествования), 
отсюда $\sigma\Omega = \sigma\tau F$
(определение $\Omega$).

\item По парадоксальности, $\sigma\tau F \subseteq \{\tau\sigma x| x \in F\}$, то есть $t \in \sigma\Omega \subseteq \{\tau\sigma x | x \in F\}$,
то есть $t = \tau\sigma\omega$ при некотором $\omega \in F$.

\item Поскольку $X$ индуктивно, то по лемме и множество $T = \{y | \tau\sigma y \in X\}$ индуктивно. По определению, ему принадлежат все фундированные элементы: $F \subseteq T$.
Значит, $\omega \in F$ влечёт $\omega \in T$, то есть $t = \tau\sigma\omega \in X$.

\end{itemize}
\end{proof}
\end{frame}

\begin{frame}{Множество $\Omega$ противоречиво}
\begin{lmm} $\Omega$ не может быть фундировано. \end{lmm}
\begin{proof} Можно показать, используя обратное включение: $\{\tau\sigma x| x \in X\} \subseteq \sigma\tau X$.\end{proof}

\begin{thm}Множество $\Omega$ не существует\end{thm}
\begin{proof}$\Omega$ фундировано, но его фундированность ведёт к противоречию.\end{proof}
\end{frame}

\begin{frame}{Построение парадокса в $U^-$}

Теперь, имея идею парадокса, покажем, что в $U^-$ есть парадоксальные универсумы.

Вспомним изоморфизм Карри-Ховарда: $P(x) \approx x \in \{ t\ |\ P(t)\} \approx X : P(x)$

После этого выразим операции $\sigma$, $\tau$ и построим $\Omega$.

\begin{enumerate}
\item $\mathcal{U} = \Pi X : \Box . ((\wp X \rightarrow X) \rightarrow \wp X)$
\item $\tau = \lambda X : \wp\mathcal{U}. \lambda A : \Box.\lambda c : (\wp A \rightarrow A).\lambda a : A.\varphi$,\\
где
 $\varphi = \Pi P^{\wp A}.(\Pi x^\mathcal{U}.X\ x\Rightarrow (P (c (\{x\ A\}\ c)))) \Rightarrow P\ a$

\item $\sigma = \lambda x : \mathcal{U} . ((x\ \mathcal{U})\ \tau)$
\end{enumerate}

Заметим, что для задания $\mathcal{U}$ нам требуются правила $(\triangle,\openbox)$.

Несложно показать, что $(\mathcal{U},\sigma,\tau)$ --- парадоксальный:
%\begin{center}
%$\mathcal{U}$ парадоксален, если для всех $X : \mathcal{U}$ выполнено $\sigma\tau X = \{\tau\sigma x | x \in X \}$
%\end{center}
$\mathcal{U}$ парадоксален, если для всех $X \in \wp\mathcal{U}$ выполнено $$\sigma\tau X = \{\tau\sigma x | x \in X \}$$
Например потому, что для всех $X: \wp \mathcal{U}$ выполнено $\sigma\tau X =_\beta \bigcap \{ P \subseteq \mathcal{U}\ |\ \forall x \in X.\tau\sigma x \in P \}$

А далее строим $\Omega$ и выписываем применение двух утверждений:
$\Omega$ фундирован и $\Omega$ фундирован $\rightarrow \bot$.
\end{frame}

\begin{frame}{Упрощённая формула}

Рассмотрим $\sigma: \mathcal{U} \rightarrow \wp\wp\mathcal{U}$ и $\tau: \wp\wp\mathcal{U}\rightarrow\mathcal{U}$.
Назовём $\mathcal{U}$ мощным универсумом, если при $C \in \wp\wp\mathcal{U}$ выполнено
$$\sigma\tau C = \{ X\ |\ \{y | \tau\sigma y \in X \} \in C\}$$

Дадим следующие обозначения:
\begin{enumerate}
\item Универсум: $$\mathcal{U} = \Pi x^\Box.(\wp\wp x \rightarrow x) \rightarrow \wp\wp x$$

\item Отображения:
       $$\tau t = \lambda X^\Box.\lambda f ^{\wp\wp X \rightarrow X}.\lambda p^{\wp X}.t \lambda x^\mathcal{U}.p (f\ (x\ X)\ f)$$
       $$\sigma s = (s\ \mathcal{U})\ \lambda t ^ {\wp\wp\mathcal{U}}.\tau t$$

\item Вспомогательное:
       $$\Delta = \lambda y ^ \mathcal{U}. (\Pi p ^ {\wp\mathcal{U}}.\sigma y p \rightarrow p \tau \sigma y) \rightarrow \bot$$

\end{enumerate}
\end{frame}

\begin{frame}{Омега и парадокс}
В качестве $\Omega$ берём $\tau \{ X | X \text{ индуктивен}\}$:
$$\Omega = \tau\ \lambda p ^ {\wp\mathcal{U}} . \Pi x ^ \mathcal{U} . \sigma x p \rightarrow p x$$

Тогда следующее выражение обитает в типе <<ложь>> ($\bot = \Pi p^\star.p$):
$$\begin{array}{c}\lambda a_0 : \Pi p : \wp \mathcal{U} . (\Pi x : \mathcal{U} . (\sigma x p) \rightarrow (p x)) \rightarrow (p \Omega)).\\
  ((a_0\ \Delta)\ \lambda x ^ {\mathcal{U}}.\lambda a_2 ^ {(\sigma x \Delta)}.\lambda a_3^ {\Pi p : \wp\mathcal{U}.(\sigma x p)\rightarrow p \tau \sigma x}.\\
  ((a_3\ \Delta)\ a_2)\ \lambda p^{\wp\mathcal{U}.(a_3\ \lambda y : \mathcal{U}.p \tau\sigma y)} (\lambda p^{\wp\mathcal{U}}.a_0\ \lambda y: \mathcal{U}.(p \tau\sigma y)))\\
  \lambda p^{\wp\mathcal{U}}.\lambda a_1 ^ {\Pi x:\mathcal{U}.\sigma x p \rightarrow p x}.(a_1\ \Omega)\ \lambda x^\mathcal{U}.a_1\ \tau\sigma x\end{array}
$$
\end{frame}

\end{document}
