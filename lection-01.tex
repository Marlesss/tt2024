\documentclass[aspectratio=169]{beamer}
\setbeamertemplate{navigation symbols}{}
\usepackage{graphicx}
\usepackage{stmaryrd}
\usepackage[T1,T2A]{fontenc}
\usepackage[utf8]{inputenc}
\usepackage[english,russian]{babel}
\usepackage{amsmath}
\usepackage{amsfonts}
\usepackage{amssymb}
\usepackage{makeidx}
\usepackage{verbatim}
\usepackage{amsthm}
\usepackage{bnf}
\usepackage{tikz}
\usepackage{enumerate}
\usepackage{mathtext}
\usepackage{mathtools}
\usepackage{mathabx}
%\usepackage[left=2cm,right=2cm,top=2cm,bottom=2cm,bindingoffset=0cm]{geometry}
\usepackage{proof}
%\usepackage{paracol}
%\usepackage{enumitem}
\usepackage{color}
\usepackage{colortbl}
\usepackage{minted}
%\usepackage{hyperref}
\usetikzlibrary{graphs}
\usetikzlibrary{graphs.standard}
\usetikzlibrary{automata,positioning}
\usepackage{float}

\begin{document}

%\theoremstyle{dfn}
\newtheorem{dfn}{Определение}[section]
\newtheorem{nte}{Замечание}[section]

\newtheorem{axiom}{Аксиома}[section]
\newtheorem{thm}{Теорема}[section]
\newtheorem{lmm}[theorem]{Лемма}
\newtheorem{statement}{Утверждение}[section]
\newtheorem{oun_paragraph}{Пункт}[section]
\newtheorem{cons}{Следствие}[section]
\newtheorem*{exm}{Пример}

\newcommand{\comb}[1]{\operatorname{\bf{\textrm{#1}}}}
\newcommand{\func}[1]{\operatorname{#1}}
\newcommand{\reduction}[1]{{\color{OrangeRed}#1}}
\newcommand{\set}[1]{\left\{#1\right\}}

\def\from#1{\par \parbox{0.7\textwidth}{\par \hfill\raggedleft \it #1}} 

\begin{frame}{}
\begin{center}\Large Лекция 1.\\ \Large Теоремы о лямбда-исчислении.\end{center}
\end{frame}

\begin{frame}{Некоторые базовые определения --- повторение}
\begin{dfn}Пред-лямбда-терм:
$$\Lambda ::= (\lambda x.\Lambda) | (\Lambda\ \Lambda) | x$$
\end{dfn}
\begin{dfn}Лямбда-терм: $\Lambda/(=_\alpha)$
\end{dfn}

\begin{dfn}$R \subseteq A \times B$ --- бинарное отношение.\\
Запишем $aRb$, если $\langle a, b \rangle \in R$\\
Отношение для инфиксной операции $a \star b$: $\langle a,b \rangle \in (\star)$
\end{dfn}
\end{frame}

\begin{frame}{$\beta$-редуцируемость}
\begin{dfn}
	$(\twoheadrightarrow_\beta)$ --- транзитивное и рефлексивное замыкание отношения $(\rightarrow_\beta)$

	А именно, будем говорить, что $A\twoheadrightarrow_{\beta}B$, если найдутся такие $X_{1}$\ldots $X_{n}$, что $A=_{\alpha}X_{1}\to_{\beta}X_{2}\to_{\beta}\ldots\to_{\beta}X_{n-1}\to_{\beta}X_{n}=_{\alpha}B$.
\end{dfn}

%$(\twoheadrightarrow_{\beta})$ "--- рефлексивное и транзитивное замыкание $(\to_{\beta})$. $(\twoheadrightarrow_{\beta})$ не обязательно приводит к нормальной форме
\begin{exm}
	$\Omega\twoheadrightarrow_{\beta}\Omega$
\end{exm}
\end{frame}

\begin{frame}{}
\begin{dfn}[Ромбовидное свойство]
	Отношение $R$ обладает ромбовидным свойством, если для любых $a,b,c$:\\
{\bf из} $aRb$, $aRc$, $b\neq{}c$ {\bf следует} существование $d$, что $bRd$ и $cRd$.
\end{dfn}

\begin{exm}
	$(\leq)$ на $\mathbb{N}_0$ обладает ромбовидным свойством: $$d = max(b,c):\quad\quad\quad 2 \leq 7, 3 \leq 7 \Rightarrow d = 7$$

	$(>)$ на $\mathbb{N}_0$ не обладает ромбовидным свойством: $$3 > 1, 3 > 0:\quad\quad\quad \text{нет } d: 1 > d, 0 > d$$
\end{exm}
\end{frame}

\begin{frame}{Теорема Чёрча-Россера}
\begin{thm}[Черча-Россера]
	$(\twoheadrightarrow_{\beta})$ обладает ромбовидным свойством.
\end{thm}


\begin{cons}
	Если у $A$ есть нормальная форма, то она единственная.
\end{cons}

\begin{proof}
	Пусть $A\twoheadrightarrow_{\beta}B$ и $A\twoheadrightarrow_{\beta}C$. $B$, $C$ "--- нормальные формы и $B\neq_{\alpha}C$. 
	Тогда по теореме Черча-Россера найдётся $D$: $B\twoheadrightarrow_{\beta}D$ и $C\twoheadrightarrow_{\beta}D$. Тогда $B=_{\alpha}D$ и $C=_{\alpha} D$ $\Rightarrow$ $B=_{\alpha}C$. Противоречие.
\end{proof}

\begin{lmm}
	Если $B$ "--- нормальная форма, то не существует $Q$ такой, что $B\to_{\beta}Q$. Значит если $B\twoheadrightarrow_{\beta}Q$, то количество шагов редукции равно 0.
\end{lmm}
\end{frame}

\begin{frame}{}
\begin{lmm}
	 \label{refl}
	Если $R$ "--- обладает ромбовидным свойством, то и $R^{*}$ (транзитивное, рефлексивное замыкание $R$) им обладает.
\end{lmm}

\begin{proof}
Две вложенных индукции.
\end{proof}
\end{frame}

%\begin{proof}
%    Пусть $M_1 R^{*} M_n$ и $M_1 R N_1$. Тогда существуют такие $M_2 \ldots M_{n-1}$, что $M_1 R M_2$ \ldots $M_{n-1} R M_n$.
%	Так как $R$ обладает ромбовидным свойством, $M_1 R M_2$ и $M_1 R N_1$, то существует такое $N_2$,
%	что $N_1 R N_2$ и $M_2 R N_2$. Аналогично, существуют такие $N_3 \ldots N_n$, что $N_{i-1} R N_{i}$ и $M_i R N_i$.
%	Мы получили такое $N_n$, что $N_1 R^{*} N_n$ и $M_n R^{*} N_n$.
%	
%	Пусть теперь $M_{1,1}R^{*}M_{1,n}$ и $M_{1,1}R^{*}M_{m,1}$, то есть имеются $M_{1,2}$\ldots$M_{1,n-1}$ и $M_{2,1}$\ldots$M_{m-1,1}$,
%	что $M_{1,i-1} R M_{1,i}$ и $M_{i-1, 1} R M_{i, 1}$.
%	Тогда существует такое $M_{2,n}$, что $M_{2,1} R^{*} M_{2,n}$ и $M_{1,n} R^{*} M_{2,n}$.
%	Аналогично, существуют такие $M_{3,n}\ldots M_{m,n}$, что $M_{i,1} R^{*} M_{i,n}$ и $M_{1,n} R^{*} M_{i,n}$.
%	Тогда $M_{1,n} R^{*} M_{m,n}$ и $M_{m,1} R^{*} M_{m,n}$.
%\end{proof}

\begin{frame}{}
\begin{lmm}
	$(\to_{\beta})$ не обладает ромбовидным свойством.
\end{lmm}

	Пусть $A=(\lambda{}x.x x)(\comb I \comb I)$. Покажем, что в таком случае не будет выполняться ромбовидное свойство:
	
	\begin{figure}[ht]
		\centering
		\begin{tikzpicture}[->, every edge/.style={draw=black,thick}]
		\node[label={\scriptsize\tikz\node[circle,draw]{$A$};}]     at (0,   0) (A)  {$(\lambda x . x x)(\comb I \comb I)$};
		\node[label={135:\scriptsize\tikz\node[circle,draw]{$B$};}] at (-2, -1) (B)  {$(\comb I \comb I)(\comb I \comb I)$};
		\node[label={45:\scriptsize\tikz\node[circle,draw]{$C$};}]  at (2,  -1) (C)  {$(\lambda x . x x)  \comb I$};
		\node[label={180:\scriptsize\tikz\node[circle,draw]{$B1$};}]                                                       at (-3, -2) (B1) {$(\comb I \comb I)\comb I$};
		\node[label={0:\scriptsize\tikz\node[circle,draw]{$B2$};}]                                                       at (-1, -2) (B2) {$\comb I(\comb I \comb I)$};
		\node[label={0:\scriptsize\tikz\node[circle,draw]{$D$};}]                                                       at (0,  -3) (D)  {$\comb I \comb I$};
		\path (A)  edge (B)
		edge (C)
		(B)  edge (B1)
		edge (B2)
		(B1) edge (D)
		(B2) edge (D)
		(C)  edge (D);
		\end{tikzpicture}
		\caption{Нет такого $D$, что $B \to_{\beta} D$ и $C \to_{\beta} D$.}
	\end{figure}	
\end{frame}

\newcommand{\bpar}{\rightrightarrows_{\beta}}

\begin{frame}{}
\begin{dfn}[Параллельная $\beta$-редукция]
	$A\bpar B$, если
	\begin{enumerate}
		\item $A=B$
		\item $A=P_{1}\ Q_{1}$, $B=P_{2}\ Q_{2}$ и $P_{1}\bpar P_{2}$, $Q_{1}\bpar Q_{2}$
		\item $A=\lambda{}x.P_{1}$, $B=\lambda{}x.P_{2}$ и 
		$P_{1}\bpar P_{2}$
		\item $A=_{\alpha}(\lambda{}x.P_1)Q_1$, $B=_{\alpha}P_2[x\coloneqq{}Q_2]$, причем $Q_2$ свободна для подстановки вместо $x$ в $P_2$ и $P_1 \bpar P_2$, $Q_1 \bpar Q_2$
	\end{enumerate}
\end{dfn}
\end{frame}

\begin{frame}{Лемма: если $P_{1}\bpar P_{2}$ и $Q_{1}\bpar Q_{2}$, то $P_{1}[x\coloneqq{}Q_{1}]\bpar P_{2}[x\coloneqq{}Q_{2}]$}
%\begin{proof}[индукция по определению ($\bpar$)]
	\begin{itemize}
		\item Пусть $P_{1}=_{\alpha}P_{2}$. Индукция по структуре выражения.
		\item Пусть $P_{1}\equiv{}A_{1}B_{1}$, $P_{2}\equiv{}A_{2}B_{2}$. По определению $(\bpar)$ $A_{1} \bpar A_{2}$ и $B_{1} \bpar B_{2}$. Тогда:
		\begin{enumerate}
			\item $x \in FV(A_{1})$. По индукционному предположению $A_{1}[x\coloneqq{}Q_{1}] \bpar A_{2}[x\coloneqq{}Q_{2}]$. Тогда $A_{1}[x\coloneqq{}Q_{1}]B_{1} \bpar A_{2}[x\coloneqq{}Q_{2}]B_{2}$. Тогда $A_{1}B_{1}[x\coloneqq{}Q_{1}] \bpar A_{2}B_{2}[x\coloneqq{}Q_{2}]$.
			\item $x \in FV(B_{1})$. По индукционному предположению $B_{1}[x\coloneqq{}Q_{1}] \bpar B_{2}[x\coloneqq{}Q_{2}]$. Тогда $A_{1}B_{1}[x\coloneqq{}Q_{1}] \bpar A_{2}B_{2}[x\coloneqq{}Q_{2}]$.
		\end{enumerate}
		\item Пусть $P_{1}\equiv{}\lambda{}y.A_{1}$, $P_{2}\equiv{}\lambda{}y.A_{2}$. По определению $(\bpar)$ $A_{1}\bpar A_{2}$. Тогда по индукционному предположению $A_{1}[x\coloneqq{}Q_{1}] \bpar A_{2}[x\coloneqq{}Q_{2}]$. Тогда 
		$\lambda{}y.(A_{1}[x\coloneqq{}Q_{1}]) \bpar \lambda{}y.(A_{2}[x\coloneqq{}Q_{2}])$ по определению $(\bpar)$. Следовательно 	$\lambda{}y.A_{1}[x\coloneqq{}Q_{1}] \bpar \lambda{}y.A_{2}[x\coloneqq{}Q_{2}]$ по определению подстановки.
		\item Пусть $P_{1}=_\alpha(\lambda{}y.A_1)B_1$, $P_{2}=_\alpha A_2[y\coloneqq{}B_2]$ и $ A_1 \bpar A_2 $, $ B_1 \bpar B_2 $. По индукционному предположению получаем, что $A_1[x \coloneqq Q_1] \bpar A_2[x \coloneqq Q_2]$, $B_1[x \coloneqq Q_1] \bpar B_2[x \coloneqq Q_2]$. 
               По определению $(\bpar)$ тогда $ (\lambda{}y.A_1[x\coloneqq Q_1])B_1[x \coloneqq Q_1] \bpar  A_2[y \coloneqq B_2][x \coloneqq Q_2]$
	\end{itemize}
%\end{proof}
\end{frame}

\begin{frame}{Лемма: $(\bpar)$ обладает ромбовидным свойством}

	Будем доказывать индукцией по определению $(\bpar)$. Покажем, что если $M \bpar M_1$ и $M \bpar M_2$, то существует ${}M_3$, что $M_1 \bpar M_3$ и $M_2 \bpar M_3$. Рассмотрим случаи:
	\begin{itemize}
		\item Если $M\equiv M_1$, то просто возьмем $M_3\equiv M_2$.
		\item Если $M\equiv \lambda{}x.P, M_1 \equiv \lambda{}x.P_1, M_2 \equiv \lambda{}x.P_2$ и $P \bpar P_1, P \bpar P_2$, то по предположению индукции 
		существует $P_3$, что $P_1 \bpar P_3, P_2 \bpar P_3$, тогда возьмем $M_3\equiv \lambda{}x.P_3$.
		\item Если $M \equiv P Q, M_1 \equiv P_1 Q_1$ --- естественное доказательство.
%и по определению $(\bpar)$ $P \bpar P_1, Q \bpar Q_1$, то рассмотрим два случая:
%		\begin{enumerate}
%			\item $M_2 \equiv P_2 Q_2$. Тогда по предположению индукции существует $P_3$, что $P_1 \bpar P_3, P_2 \bpar P_3$. Аналогично для $Q$. Тогда возьмем $M_3 \equiv P_3 Q_3$.
%			\item $P\equiv \lambda {}x.P'$ значит $P_1 \equiv \lambda{}x.P_1'$ и $ P' \bpar P_1'$. Пусть тогда $ M_2 \equiv P_2[x\coloneqq{} Q_2]$, по определению $(\bpar)$ $P' \bpar P_2, Q \bpar Q_2$. Тогда по предположению индукции и лемме $\ref{bparhelp}$ существует $M_3 \equiv P_3[x \coloneqq Q_3]$ такой, что $ P_1' \bpar P_3 $, $ Q_1 \bpar Q_3 $ и $ P_2 \bpar P_3 $, $ Q_2 \bpar Q_3 $.
%		\end{enumerate}
	\item Если $ M \equiv (\lambda{}x.P)Q $, $ M_1 \equiv P_1[x\coloneqq Q_1] $ и $ P \bpar P_1 $, $ Q \bpar Q_1$, то рассмотрим случаи:
	\begin{enumerate}
		\item $ M_2 \equiv (\lambda{}x.P_2)Q_2 $, $P \bpar P_2$, $Q \bpar Q_2$. Тогда по предположению индукции и лемме существует такой $ M_3 \equiv P_3[x \coloneqq Q_3] $, что $ P_1 \bpar P_3 $, $ Q_1 \bpar Q_3 $ и $ P_2 \bpar P_3 $, $ Q_2 \bpar Q_3 $.
		\item $ M_2 \equiv P_2[x \coloneqq Q_2]$, $ P \bpar P_2 $, $ Q \bpar Q_2 $. Тогда по предположению индукции и лемме существует такой $ M_3 \equiv P_3[x \coloneqq Q_3] $, что $ P_1 \bpar P_3 $, $ Q_1 \bpar Q_3 $ и $ P_2 \bpar P_3 $, $ Q_2 \bpar Q_3 $.
	\end{enumerate}
	\end{itemize}
\end{frame}

\begin{frame}{}
\begin{lmm}
	\
	\begin{enumerate}
		\item $(\bpar )^{*} \subseteq (\to_{\beta})^{*}$
		\item $(\to_{\beta})^{*} \subseteq (\bpar )^{*}$
	\end{enumerate}
\end{lmm}

\begin{cons}
	$(\to_{\beta})^{*}=(\bpar )^{*}$
\end{cons}

Из приведенных выше лемм и следствия докажем теорему Черча-Россера.

\begin{proof}
	$(\to_{\beta})^{*} = (\twoheadrightarrow_{\beta})$. Тогда $(\twoheadrightarrow_{\beta})=(\bpar )^{*}$. Значит из того, что $(\bpar )$ обладает ромбовидным свойством и леммы $\ref{refl}$, следует, что $(\twoheadrightarrow_{\beta})$ обладает ромбовидным свойством.
\end{proof}
\end{frame}

\begin{frame}{Нормальный и аппликативный порядок вычислений}

\begin{exm}
	Выражение $KI\Omega$ можно редуцировать двумя способами:
	\begin{enumerate}
		\item $\comb K \comb I \Omega =_{\alpha} ((\lambda{}a.\lambda{}b.a) \comb I) \Omega \to_{\beta} (\lambda{}b.\comb I)\Omega  \to_{\beta} \comb I$
		\item  $\comb K \comb I \Omega =_{\alpha} ((\lambda{}a.\lambda{}b.a) \comb I)((\lambda{}x.x \ x) (\lambda{}x.x \ x)) \twoheadrightarrow_{\beta} ((\lambda{}a.\lambda{}b.a) \comb I)((\lambda{}x.x \ x) (\lambda{}x.x \ x)) \to_{\beta} \comb K \comb I \Omega $
	\end{enumerate}
\end{exm}

%Как мы видим, в первом случае мы достигли нормальной формы, в то время как во втором мы получили бесконечную редукцию. Разница двух этих способов в порядке редукции. Первый называется нормальный порядок, а второй аппликативный. 

\begin{dfn}[нормальный порядок редукции]
	Редукция самого левого $\beta$-редекса.
\end{dfn}

\begin{dfn}[аппликативный порядок редукции]
	Редукция самого левого $\beta$-редекса из самых вложенных.
\end{dfn}

\begin{thm}[Приводится без доказательства]
	Если нормальная форма существует, она может быть достигнута нормальным порядком редукции.
\end{thm}
\end{frame}

\begin{frame}{Нормальный порядок --- медленный}

\begin{exm}
	Рассмотрим $\lambda$-выражение $(\lambda{}x.x \ x \ x \ x) (\comb I \comb I)$. Попробуем редуцировать его нормальным порядком:
	 \[(\lambda{}x.x \ x \ x \ x) (\comb I \comb I) \to_{\beta} (\comb I \comb I)(\comb I \comb I)(\comb I \comb I)(\comb I \comb I) \to_{\beta} \comb I(\comb I \comb I)(\comb I \comb I)(\comb I \comb I) \to_{\beta} (\comb I \comb I)(\comb I \comb I)(\comb I \comb I) \to_{\beta} \ldots \to_{\beta} \comb I\] 
	Как мы увидим, в данной ситуации аппликативный порядок редукции оказывается значительно эффективней: 
	\[ (\lambda{}x.x \ x \ x \ x) (\comb I \comb I) \to_{\beta} (\lambda{}x.x \ x \ x \ x) \comb I \to_{\beta} \comb I \comb I \comb I \comb I\to_{\beta} \comb I \comb I \comb I \to_{\beta} \comb I \comb I \to_{\beta} \comb I \]
\end{exm}
\end{frame}

\begin{frame}[fragile]{Как программировать? Любое значение -- замыкание}

\begin{center}\begin{tabular}{ll}
\texttt{let x = sqrt 256} &     \texttt{let x = fun () -> sqrt 256}
\end{tabular}\end{center}

Плюс мемоизация: 

\begin{center}\begin{tabular}{l}
\texttt{let x = fun () -> sqrt 256;;}\\
\texttt{let y = x;;}\\
\texttt{y () + x ()      (* вычисляется два раза *)}
\end{tabular}\end{center}

Давайте запоминать результаты!
\begin{verbatim}
type int-value = Compute of unit -> int | Result of int;;
let compute v = match !v with
    Compute f -> let res = f () in v := Result res; res
  | Result r -> r;;

let x = ref (Compute (fun () -> sqrt 256));;
let y = x;;
compute y + compute x
\end{verbatim}
\end{frame}

\begin{frame}{Ленивые и энергичные вычисления}
Энергичные вычисления: аппликативный порядок.
Ленивые вычисления: нормальный порядок + мемоизация.

If всегда ленив

\begin{center}
\texttt{let fact n = if n > 1 then n * fact (n-1) else 1}
\end{center}

Ленивое общение с внешним миром бессмысленно.
\end{frame}

\begin{frame}{Импликационный фрагмент ИИВ}

Рассмотрим подмножество ИИВ, со следующей грамматикой:
$\Phi ::= x \; | \; (\Phi \rightarrow \Phi)$

Добавим в него схему аксиом

$$\infer{\Gamma, \varphi \vdash \varphi}{}$$

	Правило введения импликации:
	\[
	\infer{\Gamma \vdash \varphi \to \psi}{\Gamma, \varphi \vdash \psi}
	\]

	Правило удаления импликации:
	\[
	\infer{\Gamma \vdash \psi}{\Gamma \vdash \varphi \to \psi && \Gamma \vdash \varphi}
	\]

\begin{thm}	
	Если $\Gamma$ и $\varphi$ состоят только из импликаций, то $\Gamma \vdash \varphi$ равносильна $\Gamma \vdash_\rightarrow \varphi$.
\end{thm}
\end{frame}

\begin{comment}
\begin{frame}{Пример}
\begin{exm}
	Докажем $\vdash \varphi \rightarrow \psi \rightarrow \varphi$
	
	\[
	\infer[(\text{Введение импликации})]
	{ \vdash \varphi \to (\psi \to \varphi) }
	{ \infer[(\text{Введение импликации})]
		{ \varphi \vdash \psi \to \varphi }
		{\varphi, \psi \vdash \varphi}
	}
	\]
\end{exm}

\begin{exm}
	Докажем $\alpha \rightarrow \beta \rightarrow \gamma, \; \alpha, \; \beta \vdash \gamma$
	
	\[
	\infer
	{ \alpha \rightarrow \beta \rightarrow \gamma, \; \alpha, \; \beta \vdash \gamma}{\infer
		{\alpha \rightarrow \beta \rightarrow \gamma, \; \alpha, \; \beta \vdash \beta \rightarrow \gamma }{\alpha \rightarrow \beta \rightarrow \gamma, \; \alpha, \; \beta \vdash \alpha \rightarrow \beta \rightarrow \gamma && \alpha \rightarrow \beta \rightarrow \gamma, \; \alpha, \; \beta \vdash \alpha} && \alpha \rightarrow \beta \rightarrow \gamma, \alpha, \; \beta \vdash \beta}
	\]
	
\end{exm}

%\begin{note}
%	В дальнейшем символом $\vdash_\rightarrow$ будем обозначать доказуемость в импликационном фрагменте.
%\end{note}
\end{frame}
\end{comment}

\begin{frame}{Замкнутость И.Ф.ИИВ}

\begin{lemma}
	\label{conj}
	Если $\Gamma \vdash \varphi$, то в любой модели Крипке из $\Vdash \Gamma$ следует $\Vdash \varphi$
\end{lemma}
\begin{proof}
	Пусть $\Gamma \vdash \varphi$. 
	Тогда $\vdash \& \Gamma \rightarrow \varphi$, где $\& \Gamma$ --- конъюнкция всех утверждений в $\Gamma$.
	По корректности моделей Крипке, будет выполнено $\Vdash \& \Gamma \rightarrow \varphi$. Переписывая $\&$ и $\rightarrow$ по определению, получаем $\Vdash \Gamma \ \Rightarrow \ \Vdash \varphi$.
\end{proof}
\end{frame}

\begin{frame}{Замкнутость И.Ф.ИИВ}
\begin{thm}
	\label{kripke}
	$\Gamma \vdash_\rightarrow \varphi$ т.и.т.т. в любой модели Крипке из $\Vdash \Gamma$ следует $\Vdash \varphi$
\end{thm}
\begin{proof}\
	\begin{itemize}
		\item $(\Rightarrow)$ Очевидно по лемме \ref{conj}.
		\item $(\Leftarrow)$ Пусть в любой модели Крипке из $\Vdash \Gamma$ следует $\Vdash \varphi$. Докажем $\Gamma \vdash_\rightarrow \varphi$.
		
		Выберем подходящую модель Крипке. Напомним, что моделью Крипке называется тройка $\langle C, \succcurlyeq, \Vdash\rangle$, где $C$ -- множество миров, $\succcurlyeq$ -- отношение частичного порядка на $C$, $\Vdash$ -- отношение вынужденности переменной.
		
		\textbf{Построим модель Крипке} $\mathcal C = \langle C, \succcurlyeq, \Vdash \rangle$. 
		
		Пусть $C = \{\Delta \ |\ \Gamma \subseteq \Delta, \Delta \text{ замкнут относительно }\vdash_\rightarrow\}$. $\Delta$ замкнут относительно доказуемости, когда для любого $\varphi$ если $\Delta \vdash_\rightarrow \varphi$, то $\varphi \in \Delta$.
		
		$C_1 \succcurlyeq C_2$, если $C_1 \supseteq C_2$.
		
		$\Delta \Vdash \alpha$, если $\alpha \in \Delta$, $\alpha$ -- переменная.
	\end{itemize}
\end{proof}
\end{frame}

\begin{frame}{Лемма: В модели $\mathcal C$ $\Delta \Vdash \varphi$ т.и.т.т. $\varphi \in \Delta$}
\begin{proof}
			\textbf{База.} $\varphi \equiv \alpha$. $\Delta \Vdash \alpha \Leftrightarrow \alpha \in \Delta$ следует из определения вынужденности.
			
			\textbf{Индукционный переход.} $\varphi \equiv \psi \rightarrow \sigma$
			
			Индукционное предположение:
			$\forall \Delta \in C: \Delta \Vdash \psi \Leftrightarrow \psi \in \Delta$, $\Delta \Vdash \sigma \Leftrightarrow \sigma \in \Delta$.
			
			Докажем, что $\Delta \Vdash \psi \rightarrow \sigma \Leftrightarrow \psi \rightarrow \sigma \in \Delta$ (два включения).
\end{proof}
\end{frame}

\begin{frame}{Доказательство $\Delta \Vdash \psi \rightarrow \sigma \Leftrightarrow \psi \rightarrow \sigma \in \Delta$}
				$(\Rightarrow)$ Пусть $\Delta \Vdash \psi \rightarrow \sigma$.
				
				Рассмотрим мир $\Pi = (\Delta \cup \{\psi\})^*$. $\Pi \Vdash \psi \rightarrow \sigma$, т.к. $\Delta \preccurlyeq \Pi$.
				
				$\psi \in \Pi$. Тогда, по инд. пред., $\Pi \Vdash \psi$. Значит, $\Pi \Vdash \sigma$. В самом деле, из определения вынужденности импликации в $\Pi$ следует, что если $\Pi \Vdash \psi$, то $\Pi \Vdash \sigma$.
				
				По инд. пред. заключаем $\sigma \in \Pi$, т.е. $\Pi \vdash_\rightarrow \sigma$, т.к. $\Pi$ -- замкнут по доказуемости. Ясно, что $\Delta, \psi \vdash_\rightarrow \sigma$. Действительно, в гипотезах доказательства $\Pi \vdash_\rightarrow \sigma$ использовалось не все бесконечное множество $\Pi$, а лишь конечный набор утверждений из него. Каждое такое утверждение выводится из $\Delta, \psi$, потому что $\Pi$ - замыкание $\Delta \cup \{\psi\}$. 
				
				Из $\Delta, \psi \vdash_\rightarrow \sigma$ следует $\Delta \vdash_\rightarrow \psi \rightarrow \sigma$. Таким образом, $\psi \rightarrow \sigma \in \Delta$.
%\end{frame}

%\begin{frame}{}
				$(\Leftarrow)$ Пусть $\psi \rightarrow \sigma \in \Delta$.
				
				Рассмотрим произвольный мир $\Pi: \Delta \preccurlyeq \Pi \land \Pi \Vdash \psi$. По инд. пред. $\psi \in \Pi$. $\psi \rightarrow \sigma \in \Pi$, т.к. $\Delta \subseteq \Pi$. $\Pi \vdash_\rightarrow \psi$, $\Pi \vdash_\rightarrow \psi \rightarrow \sigma$. Очевидно, $\Pi \vdash_\rightarrow \sigma$. $\sigma \in \Pi$. Тогда, по инд. пред., $\Pi \Vdash \sigma$. Таким образом, $\Pi \Vdash \psi \rightarrow \sigma$, а следовательно, $\Delta \Vdash \psi \rightarrow \sigma$.
\end{frame}

\begin{frame}{Завершение доказательства}
Теперь можем доказать теорему о замкнутости ИФИИВ.
\begin{proof}
	Следствие $\Gamma \vdash_\rightarrow \varphi \ \Rightarrow \ \Gamma \vdash \varphi$ очевидно.\\
	Пусть $\Gamma \vdash \varphi$. По \ref{conj} получаем, что в любой модели Крипке из $\Vdash \Gamma$ следует $\Vdash \varphi$. Отсюда, по теореме \ref{kripke}, доказывается $\Gamma \vdash_\rightarrow \varphi$.
\end{proof}
\end{frame}

\begin{frame}{Просто-типизированное лямбда-исчисление}
\begin{dfn}[$\lambda_\rightarrow$ по Карри]\vspace{-0.5cm}
%Просто-типизированное лямбда-исчисление (по Карри). %\pause Типы: $\tau ::= \alpha | (\tau\rightarrow\tau)$. \pause Язык: $\Gamma\vdash A:\varphi$
$$\infer[x \notin \Gamma]{\Gamma,x:\varphi \vdash x:\varphi}{} \quad\quad 
  \infer[x \notin \Gamma]{\Gamma\vdash \lambda x.A: \varphi\rightarrow\psi}{\Gamma,x:\varphi\vdash A:\psi} \quad\quad 
  \infer{\Gamma\vdash B A:\psi}{\Gamma\vdash A:\varphi\quad\quad\Gamma\vdash B:\varphi\rightarrow\psi}$$
\end{dfn}

\begin{dfn}[$\lambda_\rightarrow$ по Чёрчу]\vspace{-0.5cm}
$$\infer[x \notin \Gamma]{\Gamma,x:\varphi \vdash x:\varphi}{} \quad\quad 
  \infer[x \notin \Gamma]{\Gamma\vdash \lambda x^{\color{blue}\varphi}.A: \varphi\rightarrow\psi}{\Gamma,x:\varphi\vdash A:\psi} \quad\quad 
  \infer{\Gamma\vdash B A:\psi}{\Gamma\vdash A:\varphi\quad\quad\Gamma\vdash B:\varphi\rightarrow\psi}$$
\end{dfn}\pause

\begin{exm}
\begin{tabular}{l|l}
По Карри & По Чёрчу\\\hline
$\lambda f.\lambda x.f\ (f\ x) : (\alpha\rightarrow\alpha)\rightarrow(\alpha\rightarrow\alpha)$ & $\lambda f^{\alpha\rightarrow\alpha}.\lambda x^\alpha.f\ (f\ x) : (\alpha\rightarrow\alpha)\rightarrow(\alpha\rightarrow\alpha)$\\\pause
$\lambda f.\lambda x.f\ (f\ x) : (\beta\rightarrow\beta)\rightarrow(\beta\rightarrow\beta)$ & $\lambda f^{\beta\rightarrow\beta}.\lambda x^\beta.f\ (f\ x) : (\beta\rightarrow\beta)\rightarrow(\beta\rightarrow\beta)$
\end{tabular}
\end{exm}

%Изоморофизм Карри-Ховарда:\\
%Типизированы по Чёрчу: Си, Паскаль, Джава, ...\\
%Типизированы по Карри: Окамль, Хаскель, ...
\end{frame}

\begin{frame}{Теоремы о $\lambda_\rightarrow$}

\begin{lmm}[о редукции, subject reduction]
Если $A \twoheadrightarrow_\beta B$ и $\vdash A: \tau$, то $\vdash B: \tau$.
\end{lmm}

\begin{lmm}
Если $\vdash A : \tau$, то любое подвыражение $A$ также имеет тип.
\end{lmm}

\begin{thm}[Чёрча-Россера]
Если $\vdash A:\tau$, $A \twoheadrightarrow_\beta B$, $A \twoheadrightarrow_\beta C$ и $B \ne C$, то найдётся $D$, что 
$\vdash D :\tau$, и $B \twoheadrightarrow_\beta D$, $C \twoheadrightarrow_\beta D$.
\end{thm}
\end{frame}

\begin{frame}{Соответствие между исчислениями}
\begin{dfn}
$$|A|=\left\{\begin{array}{ll}x, & A = x\\
\lambda x.|Q| & A = \lambda x^\tau.Q\\
|P|\ |Q| & A = P\ Q
\end{array}\right.$$
\end{dfn}\vspace{-0.5cm}

\begin{thm}\begin{enumerate}
\item Если $\Gamma \vdash_\text{ч} A : \tau$, то $|Gamma| \vdash_\text{к} |A| : \tau$;
\item Если $\Gamma \vdash_\text{к} A : \tau$, то найдётся такой $B: A = |B|$, что $\Gamma \vdash_\text{ч} B : \tau$.
\end{enumerate}
\end{thm}

\begin{thm}[уникальность типов, для исчисления по Чёрчу]
\begin{enumerate}
\item $\Gamma \vdash_\text{ч} M : \sigma$ и $\Gamma \vdash_\text{ч} M: \tau$ влечёт $\sigma = \tau$;
\item $\Gamma \vdash_\text{ч} M : \sigma$, $\Gamma \vdash_\text{ч} N : \tau$ и $M =_\beta N$ влечёт $\sigma = \tau$.
\end{enumerate}
\end{thm}

\begin{lmm}[о расширении, subject expansion]
Если $\Gamma \vdash_\text{ч} A : \tau$ и $B \twoheadrightarrow_\beta A$, то $\Gamma \vdash_\text{ч} B : \tau$.
%Обратное для исчисления по Карри неверно: $\not\vdash K\ I\ \Omega : \tau$, но $K\ I\ \Omega \twoheadrightarrow_\beta I$ и $I : \alpha\rightarrow\alpha$.
\end{lmm}
\end{frame}

\begin{frame}{Изоморфизм Карри-Ховарда}
\begin{thm}[изоморфизм Карри-Ховарда]
\begin{enumerate}
\item Если $\Gamma\vdash\tau$, то найдётся $\Delta$, $A$, что $\Gamma = |\Delta|$ и $\Delta \vdash A : \tau$;
\item Если $\Gamma \vdash A : \tau$, то $|\Gamma| \vdash \tau$.
\end{enumerate}
\end{thm}

\end{frame}

\end{document}