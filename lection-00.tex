\documentclass[aspectratio=169]{beamer}
\usepackage{graphicx}
\usepackage{stmaryrd}
\usepackage[T1,T2A]{fontenc}
\usepackage[utf8]{inputenc}
\usepackage[english,russian]{babel}
\usepackage{amsmath}
\usepackage{amsfonts}
\usepackage{amssymb}
\usepackage{makeidx}
\usepackage{verbatim}
\usepackage{amsthm}
%\usepackage{bnf}
\usepackage{tikz}
\usepackage{enumerate}
\usepackage{mathtext}
\usepackage{mathtools}
\usepackage{mathabx}
%\usepackage[left=2cm,right=2cm,top=2cm,bottom=2cm,bindingoffset=0cm]{geometry}
\usepackage{proof}
%\usepackage{paracol}
%\usepackage{enumitem}
\usepackage{color}
\usepackage{colortbl}
%\usepackage{minted}
%\usepackage{hyperref}
\setbeamertemplate{navigation symbols}{}
\usetikzlibrary{graphs}
\usetikzlibrary{graphs.standard}
\usetikzlibrary{automata,positioning}
\usepackage{float}

\begin{document}

%\theoremstyle{dfn}
\newtheorem{dfn}{Определение}[section]
\newtheorem{nte}{Замечание}[section]

\newtheorem{axiom}{Аксиома}[section]
\newtheorem{thm}{Теорема}[section]
\newtheorem{lmm}[theorem]{Лемма}
\newtheorem{statement}{Утверждение}[section]
\newtheorem{oun_paragraph}{Пункт}[section]
\newtheorem{cons}{Следствие}[section]
\newtheorem*{exm}{Пример}

\newcommand{\comb}[1]{\operatorname{\mathcal{#1}}}
\newcommand{\func}[1]{\operatorname{#1}}
\newcommand{\reduction}[1]{{\color{OrangeRed}#1}}
\newcommand{\set}[1]{\left\{#1\right\}}

\def\from#1{\par \parbox{0.7\textwidth}{\par \hfill\raggedleft \it #1}} 

\begin{frame}{}
\begin{center}
{\LARGE Теория типов}\\\vspace{1cm}
\large О курсе
\end{center}
\end{frame}

\begin{frame}{Краткое содержание вступительного занятия}
\begin{enumerate}
\item История вопроса: что вообще изучает теория типов, типы в математике, типы в лямбда-исчислении.
Краткое повторение материала, знание которого ожидается от участников.
\item Содержание текущего курса: изоморфизм Карри-Ховарда и его применение в программировании и математике
\item Особенности преподавания.
\end{enumerate}
\end{frame}

\begin{frame}{Типы в математике}
\begin{itemize}
\item Раздел не претендует на полноту, предназначен для построения контекста.
\item Парадокс Рассела: $\{\ x\ |\ x \notin x\ \}$. Напрашивается вопрос: разве эта запись законна?
$a \in b$ --- ожидаем, что слева элемент, а справа множество. %Интуитивно хочется такие ситуации запретить.
\item A. N. Whitehead, B. Russel, Principa Mathematica (Cambridge University Press, 1910-1913) --- один из первых 
примеров формализации математики в рамках программы Гильберта, включает запрет таких записей с помощью введения типов.
Всем пропозициональным функциям (<<предикатам>>) теории присваиваем тип --- и определяем правила построения
формул с учётом типов. 
%Типы %ограничивают возможные выражения синтаксически --- в том числе 
%синтаксически запрещают парадоксальные записи.
\item По силе P.M. примерно соответствует аксиоматике Цермело, но менее удобна из-за явных типов. 
Аксиоматика ZFC явных типов не имеет,
но позволяет приписать множествам схожую характеристику (<<ранг>>).
В ZFC также невозможны множества вида $\{\ x\ |\ x \notin x\ \}$, хотя запрет не формулируется синтаксически.
\item Не пытаясь охватить всю тему про типы, сконцентрируемся только на типах в лямбда-исчислении.
\end{itemize}
\end{frame}

\begin{frame}{Лямбда-исчисление: история возникновения}

\begin{itemize}
\item Готлоб Фреге, 1893 год, <<карринг>>. Двуместную функцию $a + b$ 
можно представить как композицию двух одноместных функций: $$f(a) = \lambda x. a + x\quad\quad a + b = f(a)(b)$$
\item Моисей Шейнфинкель, 1924, комбинаторы: $$Kab = a\quad\quad Sabc = ac(bc)$$
\item Алонзо Чёрч, 1932, лямбда-исчисление: $$(\lambda x.M) = M[x := N]$$
\item Алонзо Чёрч, 1932, 1934:
формальная арифметика сложна (8 аксиом + схема аксиом индукции, исчисление предикатов...). 
В $\lambda$-исчислении арифметика выражается естественно. Попробуем $\lambda$-исчисление
расширить до логики?
\item С.Клини и Б.Россер, 1935, противоречие (модификация парадокса Ришара).
\end{itemize}
\end{frame}


\begin{frame}{Краткое изложение формализма, напоминание}

$$\Lambda ::= (\lambda x.\Lambda) | (\Lambda\ \Lambda) | x$$

Мета-язык: 
\begin{itemize}
\item Мета-переменные:\begin{itemize}
\item $A\dots Z$ --- мета-переменные для термов. 
\item $x,y,z$ --- мета-переменные для переменных. 
\end{itemize}

\item Правила расстановки скобок аналогичны правилам для кванторов:
\begin{itemize}
\item Лямбда-выражение ест всё до конца строки
\item Аппликация левоассоциативна
\end{itemize}
\end{itemize}

\begin{exm}
\begin{itemize}
\item $a\ b\ c\ (\lambda d.e\ f\ \lambda g.h)\ i \equiv \Big({\color{red}\Big(}((a\ b)\ c)\ {\color{blue}\Big(}\lambda d.((e\ f)\ (\lambda g.h)){\color{blue}\Big)}{\color{red}\Big)}\ i\Big)$
\item $0 := \lambda f.\lambda x.x;\quad(+1) := \lambda n.\lambda f.\lambda x.n\ f\ (f\ x);\quad(+2) := \lambda x.(+1)\ ((+1)\ x)$
\end{itemize}
\end{exm}
\end{frame}

\begin{comment}
\begin{frame}{Альфа-эквивалентность}
$$FV(A) = \left\{\begin{array}{ll} \{x\}, & A \equiv x\\
  FV(P)\cup FV(Q), & A \equiv P\ Q\\
  FV(P)\setminus\{x\}, & A \equiv \lambda x.P\end{array}\right.$$

Примеры: 
\begin{itemize}
\item $M := \lambda b.\lambda c.a\ c\ (b\ c)$; $FV(M) = \{a\}$
\item $N := x\ (\lambda x.(x\ (\lambda y.x)))$; $FV(N) = \{x\}$
\end{itemize}

\begin{dfn}$A=_\alpha B$, если и только если выполнено одно из трёх:
\begin{enumerate}
\item $A \equiv x$, $B \equiv y$, $x \equiv y$;
\item $A \equiv P_a Q_a$, $B \equiv P_b Q_b$ и $P_a =_\alpha P_b$, $Q_a =_\alpha Q_b$;
\item $A \equiv (\lambda x.P)$, $B \equiv (\lambda y.Q)$, $P[x := t] =_\alpha Q[y := t]$, где $t$ не входит в $A$ и $B$.
\end{enumerate}\end{dfn}

\begin{dfn}$L = \Lambda/=_\alpha$\end{dfn}
\end{frame}
\end{comment}

\begin{frame}{Альфа-эквивалентность}
\begin{enumerate}
\item $A \equiv x$, $B \equiv y$, $x \equiv y$;
\item $A \equiv P_a Q_a$, $B \equiv P_b Q_b$ и $P_a =_\alpha P_b$, $Q_a =_\alpha Q_b$;
\item $A \equiv (\lambda x.P)$, $B \equiv (\lambda y.Q)$, $P[x := t] =_\alpha Q[y := t]$, где $t$ не входит в $A$ и $B$.
\end{enumerate}\vspace{-0.2cm}\begin{lmm}\vspace{-0.3cm}
$$\lambda a.\lambda b.a\ b =_\alpha \lambda b.\lambda a.b\ a$$\vspace{-0.3cm}
\end{lmm}\vspace{-0.3cm}
\begin{proof}
\begin{center}\begin{tabular}{rcll}
  $t$ & $=_\alpha$ &$t$& Правило 1\\
  $s$ & $=_\alpha$ &$s$& Правило 1\\
  $t\ s$ & $=_\alpha$ &$t\ s$& Правило 2\\
  $\lambda b.(t\ b)$ & $=_\alpha$ &$\lambda a.(t\ a)$ & Правило 3\\
  $\lambda a.\lambda b.(a\ b)$ & $=_\alpha$ &$\lambda b.\lambda a.(b\ a)$ & Правило 3\\
\end{tabular}\end{center}\vspace{-0.3cm}
\end{proof}
\end{frame}

\begin{frame}{Бета-редукция}
%Интуиция: вызов функции.
%\begin{center}\begin{tabular}{l|l}
%$\lambda$-выражение & Python \\\hline
%$\lambda f.\lambda x.f\ x$ & \texttt{def one(f,x): return f(x)}\\
%$(\lambda x.x\ x)\ (\lambda x.x\ x)$ & \texttt{(lambda x: x x) (lambda x: x x)}\\
% &   \texttt{def omega(x): return x(x); omega(omega)}
%\end{tabular}\end{center}

%\pause
\begin{dfn} Терм вида $(\lambda x.P)\ Q$ --- бета-редекс.\end{dfn}
\begin{dfn} $A \rightarrow_\beta B$, если:
\begin{enumerate}
\item $A \equiv (\lambda x.P)\ Q$, $B \equiv P\ [x := Q]$, при условии свободы для подстановки;
\item $A \equiv (P\ Q)$, $B \equiv (P'\ Q')$, при этом $P \rightarrow_\beta P'$ и $Q = Q'$, либо $P = P'$ и $Q \rightarrow_\beta Q'$;
\item $A \equiv (\lambda x.P)$, $B \equiv (\lambda x.P')$, и $P \rightarrow_\beta P'$.
\end{enumerate}
\end{dfn}%\end{frame}

%\begin{frame}{Бета-редукция, пример}
\begin{exm}
$(\lambda x.x\ x)\ (\lambda n.n) \rightarrow_\beta (\lambda n.n)\ (\lambda n.n) \rightarrow_\beta \lambda n.n$
\end{exm}
\begin{exm}
$(\lambda x.x\ x)\ (\lambda x.x\ x) \rightarrow_\beta (\lambda x.x\ x)\ (\lambda x.x\ x)$
\end{exm}
\end{frame}

%\begin{frame}{Нормальная форма}
%\begin{dfn}Лямбда-терм $N$ находится в нормальной форме, если нет $Q$: $N \rightarrow_\beta Q$.\end{dfn}
%\begin{exm}В нормальной форме:\\
%$\lambda f.\lambda x.x\ (f\ (f\ \lambda g.x))$\end{exm}\pause
%\begin{exm}Не в нормальной форме (редексы подчёркнуты):\\
%$\lambda f.\lambda x.\underline{(\lambda g.x)\ (f\ (f\ x))}$\\
%$(\underline{(\lambda x.x)\ (\lambda x.x)})\ (\underline{(\lambda x.x)\ (\lambda x.x)})$
%\end{exm}
%\begin{dfn}$(\twoheadrightarrow_\beta)$ --- транзитивное и рефлексивное замыкание $(\rightarrow_\beta)$.\end{dfn}
%\end{frame}

%\end{frame}

\begin{frame}{Бета-эквивалентность, неподвижная точка}
%\begin{exm}$\Omega = (\lambda x.x\ x)\ (\lambda x.x\ x)$ не имеет нормальной формы:
%$\Omega \rightarrow_\beta \Omega$\end{exm}
\begin{dfn}$(=_\beta)$ --- транзитивное, рефлексивное и симметричное замыкание $(\rightarrow_\beta)$.\end{dfn}

\begin{thm}[следствие из теоремы Чёрча-Россера]
Если $A =_\beta B$, то найдётся $C$, что $A \twoheadrightarrow_\beta C$ и $B \twoheadrightarrow_\beta C$.
\end{thm}
То есть, можно интуитивно воспринимать $A =_\beta B$ как вычислительное равенство, равенство результатов.

\begin{thm}Для любого терма $N$ найдётся такой терм $R$, что $R =_\beta N\ R$.\end{thm}
\begin{proof}Пусть $Y = \lambda f.(\lambda x.f\ (x\ x))\ (\lambda x.f\ (x\ x))$.
Тогда $R := Y\ N$:

$$Y\ N =_\beta (\lambda x.N\ ({\color{red}x}\ {\color{blue}x}))\ (\lambda x.N\ (x\ x)) =_\beta N\ ({\color{red}(\lambda x.N\ (x\ x)})\ ({\color{blue}\lambda x.N\ (x\ x)}))$$
\end{proof}
\end{frame}
%\end{comment}

\begin{frame}{Подробнее о лямбда-исчислении как логике}

<<Анахроническое>> изложение, пересказ современным языком.
\begin{itemize}
\item В лямбда-исчисление введём логический символ $\rightarrow$. Все формулы исчисления будем считать
логическими высказываниями.
Добавим логические аксиомы.

Ожидаем такое: $\vdash 0+1 = 1$

\item Получение противоречия: определим минимальные требования к исчислению. 
Очевидно, хотя бы следующее мы должны уметь доказывать:

\begin{enumerate}
	\item $\vdash A \rightarrow A$
	\item $\vdash (A \rightarrow (A \rightarrow B)) \rightarrow (A \rightarrow B)$
	\item Если $\vdash A$ и $\vdash A \rightarrow B$, то $\vdash B$.
\end{enumerate}

\item Менее очевидное: $\vdash A \rightarrow B$, если $A =_{\beta} B$. Мотивация: если $0+1 =_\beta 1$, то
$X(0+1)$ всегда можно заменить на $X(1)$, следует из определения равенства по Лейбницу.

\item Заметим: $0+1 \twoheadrightarrow_\beta 1$, поэтому $\vdash (1 =1) \rightarrow (0+1 = 1)$.
Из чего уже видны возможные контуры исчисления.
\end{itemize}

\end{frame}

\begin{frame}{Парадокс Карри}

$\Phi_{\alpha} := Y\;(\lambda x.x \rightarrow \alpha)$

Редуцируя $\Phi_{\alpha}$, получаем:

%\begin{align*}
$$\Phi_{\alpha}  \twoheadrightarrow_\beta (\lambda x.x \rightarrow \alpha)\;(Y\;(\lambda x.x \rightarrow \alpha)) \twoheadrightarrow_\beta\Phi_{\alpha} \rightarrow \alpha$$
%\end{align*}

И доказательство:

\begin{tabular}{ll}
	1) $\Phi_\alpha\rightarrow(\Phi_\alpha\rightarrow\alpha)$ & $(A\rightarrow A)$ и $\Phi_{\alpha} =_{\beta} \Phi_{\alpha} \rightarrow \alpha$\\
	2) $(\Phi_\alpha\rightarrow\Phi_\alpha\rightarrow\alpha)\rightarrow(\Phi_\alpha\rightarrow\alpha)$ & Так как $(A \rightarrow (A \rightarrow B)) \rightarrow (A \rightarrow B)$\\
	3) $\Phi_\alpha\rightarrow\alpha$ & MP 1, 2\\
	4) $(\Phi_\alpha \rightarrow \alpha) \rightarrow \Phi_\alpha$ & $(A\rightarrow A)$ и $\Phi_\alpha \rightarrow \alpha =_{\beta} \Phi_\alpha$\\
	5) $\Phi_\alpha$ & MP 3, 4\\
	6) $\alpha$ & MP 5, 3
\end{tabular}

Собственно парадокс Карри: <<Если данное высказывание верно, то луна сделана из зелёного сыра>>. То есть,
$$\Phi_\alpha \leftrightarrow (\Phi_\alpha\rightarrow\alpha)$$

\end{frame}

\begin{frame}{Лямбда-исчисление как вычислительная модель}
\begin{itemize}
\item Из исчисления А. Чёрч выделил некоторую часть и доказал её непротиворечивость:
Church, A. (1935). “A Proof of Freedom from Contradiction.” Proceedings of the National Academy of Sciences of the United States of America, 21(5):275–281.
\item Но затем предложил смотреть на исчисление как на вычислительную модель:
Church, A. (1936). “An Unsolvable Problem of Elementary Number Theory.” American Journal of Mathematics, 58(2):345–363, 1936.
\item Начала современного понимания теории типов были заложены в этой работе:
Church, A. (1940). A formulation of the simple theory of types, Journal of Symbolic Logic 5, pp. 56–68.
\item Применение типов в лямбда-исчислении позволяет достичь схожих результатов с Principa Mathematica: 
синтаксическое ограничение допустимых формул, исключение части формул из исчисления.
Мы начнём с краткого повторения просто-типизированного лямбда-исчисления и покажем невыразимость в нём Y-комбинатора.
%\item Впоследствии мы докажем, что просто-типизированное лямбда-исчисление сильно нормализуемо --- 
%у любой формулы есть нормальная форма, и отсутствует возможность для бесконечной редукции формул. 
\end{itemize}
\end{frame}

\begin{frame}{Импликационный фрагмент ИИВ}

Рассмотрим подмножество ИИВ, со следующей грамматикой:

$\Phi ::= x \; | \; \Phi \rightarrow \Phi \; | \; (\Phi)$

Добавим в него схему аксиом

$$\infer{\Gamma, \varphi \vdash \varphi}{}$$

	Правило введения импликации:
	\[
	\infer{\Gamma \vdash \varphi \to \psi}{\Gamma, \varphi \vdash \psi}
	\]

	Правило удаления импликации:
	\[
	\infer{\Gamma \vdash \psi}{\Gamma \vdash \varphi \to \psi && \Gamma \vdash \varphi}
	\]

\begin{theorem}	
	Если $\Gamma$ и $\varphi$ состоят только из импликаций, то $\Gamma \vdash \varphi$ равносильна $\Gamma \vdash_\rightarrow \varphi$.
\end{theorem}
\end{frame}

\begin{frame}{Пример}
\begin{exm}
	Докажем $\vdash \varphi \rightarrow \psi \rightarrow \varphi$
	
	\[
	\infer[(\text{Введение импликации})]
	{ \vdash \varphi \to (\psi \to \varphi) }
	{ \infer[(\text{Введение импликации})]
		{ \varphi \vdash \psi \to \varphi }
		{\varphi, \psi \vdash \varphi}
	}
	\]
\end{exm}

\begin{exm}
	Докажем $\alpha \rightarrow \beta \rightarrow \gamma, \; \alpha, \; \beta \vdash \gamma$
	
	\[
	\infer
	{ \alpha \rightarrow \beta \rightarrow \gamma, \; \alpha, \; \beta \vdash \gamma}{\infer
		{\alpha \rightarrow \beta \rightarrow \gamma, \; \alpha, \; \beta \vdash \beta \rightarrow \gamma }{\alpha \rightarrow \beta \rightarrow \gamma, \; \alpha, \; \beta \vdash \alpha \rightarrow \beta \rightarrow \gamma && \alpha \rightarrow \beta \rightarrow \gamma, \; \alpha, \; \beta \vdash \alpha} && \alpha \rightarrow \beta \rightarrow \gamma, \alpha, \; \beta \vdash \beta}
	\]
	
\end{exm}

%\begin{note}
%	В дальнейшем символом $\vdash_\rightarrow$ будем обозначать доказуемость в импликационном фрагменте.
%\end{note}
\end{frame}

\begin{comment}
\begin{frame}{Замкнутость И.Ф.ИИВ}

\begin{lemma}
	\label{conj}
	Если $\Gamma \vdash \varphi$, то в любой модели Крипке из $\Vdash \Gamma$ следует $\Vdash \varphi$
\end{lemma}
\begin{proof}
	Пусть $\Gamma \vdash \varphi$. 
	Тогда $\vdash \& \Gamma \rightarrow \varphi$, где $\& \Gamma$ --- конъюнкция всех утверждений в $\Gamma$.
	По корректности моделей Крипке, будет выполнено $\Vdash \& \Gamma \rightarrow \varphi$. Переписывая $\&$ и $\rightarrow$ по определению, получаем $\Vdash \Gamma \ \Rightarrow \ \Vdash \varphi$.
\end{proof}
\end{frame}

\begin{frame}{Замкнутость И.Ф.ИИВ}
\begin{theorem}
	\label{kripke}
	$\Gamma \vdash_\rightarrow \varphi$ т.и.т.т. в любой модели Крипке из $\Vdash \Gamma$ следует $\Vdash \varphi$
\end{theorem}
\begin{proof}\
	\begin{itemize}
		\item $(\Rightarrow)$ Очевидно по лемме \ref{conj}.
		\item $(\Leftarrow)$ Пусть в любой модели Крипке из $\Vdash \Gamma$ следует $\Vdash \varphi$. Докажем $\Gamma \vdash_\rightarrow \varphi$.
		
		Выберем подходящую модель Крипке. Напомним, что моделью Крипке называется тройка $\langle C, \succcurlyeq, \Vdash\rangle$, где $C$ -- множество миров, $\succcurlyeq$ -- отношение частичного порядка на $C$, $\Vdash$ -- отношение вынужденности переменной.
		
		\textbf{Построим модель Крипке} $\mathcal C = \langle C, \succcurlyeq, \Vdash \rangle$. 
		
		Пусть $C = \{\Delta \ |\ \Gamma \subseteq \Delta, \Delta \text{ замкнут относительно }\vdash_\rightarrow\}$. $\Delta$ замкнут относительно доказуемости, когда для любого $\varphi$ если $\Delta \vdash_\rightarrow \varphi$, то $\varphi \in \Delta$.
		
		$C_1 \succcurlyeq C_2$, если $C_1 \supseteq C_2$.
		
		$\Delta \Vdash \alpha$, если $\alpha \in \Delta$, $\alpha$ -- переменная.		
	\end{itemize}
\end{proof}
\end{frame}

\begin{frame}{Лемма: В модели $\mathcal C$ $\Delta \Vdash \varphi$ т.и.т.т. $\varphi \in \Delta$}
			\textbf{База.} $\varphi \equiv \alpha$. $\Delta \Vdash \alpha \Leftrightarrow \alpha \in \Delta$ следует из определения вынужденности.
			
			\textbf{Индукционный переход.} $\varphi \equiv \psi \rightarrow \sigma$
			
			Индукционное предположение:
			$\forall \Delta \in C: \Delta \Vdash \psi \Leftrightarrow \psi \in \Delta$, $\Delta \Vdash \sigma \Leftrightarrow \sigma \in \Delta$.
			
			Докажем, что $\Delta \Vdash \psi \rightarrow \sigma \Leftrightarrow \psi \rightarrow \sigma \in \Delta$.
				
\end{frame}

\begin{frame}{}
				$(\Rightarrow)$ Пусть $\Delta \Vdash \psi \rightarrow \sigma$.
				
				Рассмотрим мир $\Pi = (\Delta \cup \{\psi\})^*$. $\Pi \Vdash \psi \rightarrow \sigma$, т.к. $\Delta \preccurlyeq \Pi$.
				
				$\psi \in \Pi$. Тогда, по инд. пред., $\Pi \Vdash \psi$. Значит, $\Pi \Vdash \sigma$. В самом деле, из определения вынужденности импликации в $\Pi$ следует, что если $\Pi \Vdash \psi$, то $\Pi \Vdash \sigma$.
				
				По инд. пред. заключаем $\sigma \in \Pi$, т.е. $\Pi \vdash_\rightarrow \sigma$, т.к. $\Pi$ -- замкнут по доказуемости. Ясно, что $\Delta, \psi \vdash_\rightarrow \sigma$. Действительно, в гипотезах доказательства $\Pi \vdash_\rightarrow \sigma$ использовалось не все бесконечное множество $\Pi$, а лишь конечный набор утверждений из него. Каждое такое утверждение выводится из $\Delta, \psi$, потому что $\Pi$ - замыкание $\Delta \cup \{\psi\}$. 
				
				Из $\Delta, \psi \vdash_\rightarrow \sigma$ следует $\Delta \vdash_\rightarrow \psi \rightarrow \sigma$. Таким образом, $\psi \rightarrow \sigma \in \Delta$.
\end{frame}

\begin{frame}{}
				$(\Leftarrow)$ Пусть $\psi \rightarrow \sigma \in \Delta$.
				
				Рассмотрим произвольный мир $\Pi: \Delta \preccurlyeq \Pi \land \Pi \Vdash \psi$. По инд. пред. $\psi \in \Pi$. $\psi \rightarrow \sigma \in \Pi$, т.к. $\Delta \subseteq \Pi$. $\Pi \vdash_\rightarrow \psi$, $\Pi \vdash_\rightarrow \psi \rightarrow \sigma$. Очевидно, $\Pi \vdash_\rightarrow \sigma$. $\sigma \in \Pi$. Тогда, по инд. пред., $\Pi \Vdash \sigma$. Таким образом, $\Pi \Vdash \psi \rightarrow \sigma$, а следовательно, $\Delta \Vdash \psi \rightarrow \sigma$.
\end{frame}

\begin{frame}{Завершение доказательства}
Теперь можем доказать теорему о замкнутости ИФИИВ.
\begin{proof}
	Следствие $\Gamma \vdash_\rightarrow \varphi \ \Rightarrow \ \Gamma \vdash \varphi$ очевидно.\\
	Пусть $\Gamma \vdash \varphi$. По \ref{conj} получаем, что в любой модели Крипке из $\Vdash \Gamma$ следует $\Vdash \varphi$. Отсюда, по теореме \ref{kripke}, доказывается $\Gamma \vdash_\rightarrow \varphi$.
\end{proof}
\end{frame}
\end{comment}

\begin{frame}{Просто типизированное по Карри $\lambda$-исчисление}

\begin{dfn}
	Тип в просто типизированном $\lambda$-исчислении по Карри --- это либо маленькая греческая буква ($\alpha, \phi, \theta, \ldots$), либо импликация ($\theta_1 \rightarrow \theta_2$)
	
	Таким образом, $\Theta ::= \theta_{i} | (\Theta \rightarrow \Theta)$
	
	Импликация при этом считается правоассоциативной операцией.
\end{dfn}

\begin{dfn}
	Язык просто типизированного $\lambda$-исчисления --- это язык бестипового $\lambda$-исчисления.
\end{dfn}

\begin{dfn}
	Контекст $\Gamma$ --- это список выражений вида $A: \theta$, где $A$ --- $\lambda$-терм, а $\theta$ --- тип.
\end{dfn}
\end{frame}

\begin{frame}{Исчисление по Карри}
	Аксиома
	$$\infer[\text{если } x \text{ не входит в } \Gamma]{\Gamma, x : \theta \vdash x : \theta}{}$$
	
		Правило типизации абстракции
		\[
		\infer[\text{если } x \text{ не входит в } \Gamma]{\Gamma \vdash (\lambda \; x. \; P) : \varphi \rightarrow \psi}{\Gamma, x : \varphi \vdash P : \psi}
		\]
		Правило типизации аппликации:
		\[
		\infer{\Gamma \vdash PQ : \psi}{\Gamma \vdash P : \varphi \to \psi && \Gamma \vdash Q : \varphi}
		\]
        Мы допускаем в исчислении только те лямбда-выражения, которые имеют тип.
\end{frame}

\begin{frame}{Пример}
	Докажем $\vdash \lambda \; x. \; \lambda \; y. \; x : \alpha \rightarrow \beta \rightarrow \alpha$
	
	\[
	\infer[(\text{Правило типизации абстракции})]
	{ \vdash \lambda \; x. \; \lambda \; y. \; x : \alpha \rightarrow \beta \rightarrow \alpha }
	{ \infer[(\text{Правило типизации абстракции})]
		{x: \alpha \vdash \lambda \; y. \; x : \beta \rightarrow \alpha}
		{x: \alpha, y : \beta \vdash x : \alpha}
	}
	\]
\end{frame}

\begin{frame}{Пример}
	Докажем $\vdash \lambda \; x. \; \lambda \; y. \; x \; y : (\alpha \rightarrow \beta) \rightarrow \alpha \rightarrow \beta$
	
	\[
	\infer
	{\vdash \lambda \; x. \; \lambda \; y. \; x \; y : (\alpha \rightarrow \beta) \rightarrow \alpha \rightarrow \beta}
	{
		\infer
		{x: \alpha \rightarrow \beta \vdash \lambda \; y. \; x \; y : \alpha \rightarrow \beta}
		{
			\infer
			{x: \alpha \rightarrow \beta, y : \alpha \vdash x \; y : \beta}
			{x: \alpha \rightarrow \beta, y : \alpha \vdash x: \alpha \rightarrow \beta && x: \alpha \rightarrow \beta, y : \alpha \vdash y: \alpha}
		}
	}
	\]
\end{frame}

\begin{frame}{Отсутствие типа у Y-комбинатора}

\begin{thm}
	$Y$-комбинатор не типизируется в просто типизированном по Карри $\lambda$-исчислении.
\end{thm}

%$Y \; f =_{\beta} f \; (Y \; f)$, поэтому $Y \; f$ и $f \; (Y \; f)$ должны иметь одинаковые типы.

%Пусть $Y \; f : \alpha$

%Тогда $Y : \beta \rightarrow \alpha, f : \beta$

%Из $f \; (Y \; f) : \alpha$ получаем $f: a \rightarrow \alpha$ (так как $Y f : \alpha$)

%Тогда $\beta = \alpha \rightarrow \alpha$, из этого получаем $Y : (\alpha \rightarrow \alpha) \rightarrow \alpha$

%Можно доказать, что $\lambda \; x. \; x : \alpha \rightarrow \alpha$. Тогда $Y \; \lambda \; x. \; x : \alpha$, то есть любой тип является обитаемым. Так как это невозможно, $Y$-комбинатор не может иметь типа, так как тогда он сделает нашу логику противоречивой.

%\subparagraph{Формальное доказательство}

Докажем от противного. Есть вывод типа выражения $x \; x$. Единственный вариант типизации:

$$\infer{\Gamma \vdash x x: \psi}{\Gamma \vdash x: \varphi \rightarrow \psi && \Gamma \vdash x: \varphi }$$

Рассмотрим типизацию $\Gamma \vdash x: \varphi \rightarrow \psi$ и $\Gamma \vdash x: \varphi$. 
$x$ типизируется неизбежно следующим образом.

$$\infer{\Gamma', x: \varphi \rightarrow \psi, x: \varphi \vdash x x: \psi}{\Gamma', x: \varphi \rightarrow \psi, x: \varphi \vdash x: \varphi \rightarrow \psi && \Gamma', x: \varphi \rightarrow \psi, x: \varphi \vdash x: \varphi }$$

В $\Gamma$ переменная $x$ встречается два раза.
\end{frame}

\begin{frame}{Изоморфизм Карри-Ховарда}

\begin{tabular}{ | p{7cm} | p{7cm} | }
	\hline
	Просто типизированное $\lambda$-исчисление & Импликативный фрагмент ИИВ \\ \hline
	$\Gamma, x : \theta \vdash x : \theta$ & $\Gamma, \varphi \vdash \varphi$ \\
	&\\
	$\infer{\Gamma \vdash (\lambda \; x. \; P) : \varphi \rightarrow \psi}{\Gamma, x : \varphi \vdash P : \psi}$ & $\infer{\Gamma \vdash \varphi \to \psi}{\Gamma, \varphi \vdash \psi}$  \\
	&\\
	$\infer{\Gamma \vdash PQ : \psi}{\Gamma \vdash P : \varphi \to \psi && \Gamma \vdash Q : \varphi}$ & $\infer{\Gamma \vdash \psi}{\Gamma \vdash \varphi \to \psi && \Gamma \vdash \varphi}$ \\
	\hline
\end{tabular}

$\newline$
\begin{tabular}{ | p{7cm} | p{7cm} | }
	\hline
	Просто типизированное $\lambda$-исчисление & Импликативный фрагмент ИИВ \\ \hline
	Тип & Высказывание \\
	Терм & Доказательство высказывания  \\
	Проверка типа & Проверка доказательства на корректность \\
	Обитаемый тип & Доказуемое высказывание \\
	\hline
\end{tabular}

\end{frame}

\begin{frame}{Часть 2. Анонс содержания курса}
\begin{itemize}
\item Формальный материал до этого момента должен быть всем знаком --- 
выше были соображения по поводу истории вопроса и напоминание общих определений. Если встретилось 
что-то неизвестное, задавайте вопросы сами.
\item С этого момента начнётся нечто новое, сейчас будут анонсированы основные темы курса.
\item Курс строится вокруг изоморфизма Карри-Ховарда --- изоморфизма между матлогом и лямбда-исчислением.
Он состоит из обсуждения двух частей: что мы можем получить из матлога для программирования, 
и что можем получить для матлога из программирования.
\end{itemize}
\end{frame}

\begin{frame}[fragile]{$(\Rightarrow)$: изучение языков программирования}
\begin{itemize}
\item Малая выразительная сила просто-типизированного лямбда исчисления (полиномы).
\item Добавим дополнительные конструкции (например, кванторы).
\item Логика первого порядка: зависимые типы.
Какой тип у \verb!sprintf!?
\begin{center}
\begin{verbatim} 
sprintf "%d" : int -> string
sprintf "%d + %d" : int*int -> string
\end{verbatim}
\end{center}

Например, Идрис позволяет тип выписать.

\item Логика второго порядка: генерики.

\begin{center}
\begin{tabular}{l|l}
Программа & Доказывает\\\hline
\verb!let id x = x! & $\forall x.x \rightarrow x$
\end{tabular}\end{center}

\item Классические функциональные языки --- типовая система Хиндли-Милнера 
(разрешимый вариант системы F, соответствующей логике второго порядка).

\item Алгоритмы вывода типов, анализ и верификация программ --- используют матлог.
\end{itemize}
\end{frame}

\begin{frame}{$(\Leftarrow)$: изучение логики и доказательств через написание программ}
\begin{itemize}
\item Пер Мартин-Лёф, Intuonistic Type Theory: версии 1972 и 1979.
\item Множество расширений и вариантов.
\item Такие инструменты как Coq, Agda, Lean используют варианты этой теории.
\item Мы будем рассматривать некоторую родственную теорию, гомотопическую теорию типов.
\end{itemize}
\end{frame}

\begin{frame}[fragile]{Гомотопическая теория типов}
\begin{itemize}
\item Центральный вопрос --- что такое равенство.
\item Классический матлог: это предикат, удовлетворяющий свойствам.
\item Однако, свойства обычно слишком общие (класс эквивалентности?). Интуитивно хочется большего,
равенство не всегда просто эквивалентность.
\item Изоморфизм Карри-Ховарда-Воеводского:
\begin{tabular}{lll}
Логика & $\lambda$-исчисление & Топология\\\hline
Утверждение & Тип & Пространство \\
Доказательство & Значение & Точка в пространстве\\
Предикат $(=)$ & Зависимый тип $(=)$ & Путь между точками
\end{tabular}
\item Реализация: кубическая теория типов, Аренд.
\end{itemize}
\end{frame}

\begin{frame}{Часть 3. Построение курса}
\begin{enumerate}
\item Аналогично с матлогом, будет разделение на теорию и практику.
\item Теория: знание определений, идей, теорем.
\item Практика: лабы на доказательства теорем с использованием языка Аренд, возможны дополнительные околокомпиляторные лабы.
\item Для закрытия предмета надо набрать баллы практическими заданиями и сдать зачёт/экзамен.
\end{enumerate}
\end{frame}

\end{document}
