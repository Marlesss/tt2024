\documentclass[10pt,a4paper,oneside]{article}
\usepackage[utf8]{inputenc}
\usepackage[english,russian]{babel}
\usepackage{amsmath}
\usepackage{amsthm}
\usepackage{amssymb}
\usepackage{enumerate}
\usepackage{stmaryrd}
\usepackage{comment}
\usepackage{cmll}
\usepackage{mathrsfs}
\usepackage{hyperref}
\usepackage[left=2cm,right=2cm,top=2cm,bottom=2cm,bindingoffset=0cm]{geometry}
\usepackage{proof}
\usepackage{tikz}
\usepackage{multicol}

\makeatletter
\newcommand{\dotminus}{\mathbin{\text{\@dotminus}}}

\newcommand{\@dotminus}{%
  \ooalign{\hidewidth\raise1ex\hbox{.}\hidewidth\cr$\m@th-$\cr}%
}
\makeatother

\usetikzlibrary{arrows,backgrounds,patterns,matrix,shapes,fit,calc,shadows,plotmarks}

\newtheorem{definition}{Определение}
\begin{document}

\begin{center}{\Large\textsc{\textbf{Теоретические домашние задания}}}\\
             \it Теория типов, ИТМО, совместно М3232-М3239 и M3332-M3339, весна 2024 года\end{center}

\section*{Домашнее задание №1: лямбда исчисление --- бестиповое и просто-типизированное}

\begin{enumerate}
\item Бесконечное количество комбинаторов неподвижной точки. Дадим следующие определения
$$\begin{array}{l}
L := \lambda abcdefghijklmnopqstuvwxyzr.r(thisisafixedpointcombinator)\\
R := LLLLLLLLLLLLLLLLLLLLLLLLLL\end{array}$$
В данном определении терм $R$ является комбинатором неподвижной точки: каков бы ни был терм
$F$, выполнено $R\ F =_\beta F\ (R\ F)$.
\begin{enumerate}
\item Докажите, что данный комбинатор --- действительно комбинатор неподвижной точки.
\item Пусть в качестве имён переменных разрешены русские буквы. Постройте аналогичное выражение
по-русски: с 33 параметрами и осмысленной русской фразой в терме $L$; покажите, что оно является
комбинатором неподвижной точки.
\end{enumerate}

\item Найдите необитаемый тип в просто-типизированном лямбда-исчислении (напомним: тип $\tau$ 
называется необитаемым, если ни для какого $P$ не выполнено $\vdash P : \tau$).

\item Напомним определение: комбинатор --- лямбда-выражение без свободных переменных. Также напомним:
$$\begin{array}{l}
S := \lambda x.\lambda y.\lambda z.x\ z\ (y\ z)\\
K := \lambda x.\lambda y.x\\
I := \lambda x.x
\end{array}$$

Известна теорема о том, что для любого комбинатора $X$ можно найти выражение $P$
(состоящее только из скобок, пробелов и комбинаторов $S$ и $K$), что $X =_\beta P$.
Будем говорить, что комбинатор $P$ \emph{выражает} комбинатор $X$ в базисе $SK$.

Косвенным аргументом (пояснением, но не доказательством!) в пользу этой теоремы являются 
два следующих соображения:
\begin{itemize}
\item теорема о замкнутости ИфИИВ: если $\vdash \varphi$, то $\vdash_\rightarrow \varphi$,
значит, если выражение имеет тип, то этот тип можно получить с помощью доказательства в
стиле Гильберта;
\item типы комбинаторов $S$ и $K$ --- это, соответственно, вторая и первая схемы аксиом.
\end{itemize}

Докажите тип следующих выражений как логическое высказывание с помощью
гильбертового вывода и, пользуясь этим доказательством как источником вдохновения, 
выразите комбинаторы в базисе $SK$:
\begin{enumerate}
\item $\lambda x.\lambda y.\lambda z.y$
\item $\lambda x.\lambda y.\lambda z.y x z$
\item $\overline{1}$
\item $Not$
\item $Xor$
\item $InR$
\end{enumerate}

\item Покажите на основании следующего преобразования полноту комбинаторного базиса SKI
(проведите полное рассуждение по индукции, из которого будет следовать отсутствие 
в результате других термов, кроме SKI, бета-эквивалентность
и определённость результата для всех комбинаторов $\sigma$):

$$[\sigma]=\left\{\begin{array}{ll}
x, & \sigma = x\\
\left[\varphi\right]\ [\psi], & \sigma = \varphi\ \psi\\
K\ [\varphi], & \sigma = \lambda x.\varphi,\quad x \notin FV(\varphi)\\
I, & \sigma = \lambda x.x \\
\left[\lambda x.\left[\lambda y.\varphi\right]\right], & \sigma = \lambda x.\lambda y.\varphi,\quad x \in FV(\varphi), x \ne y\\
S\ [\lambda x.\varphi]\ [\lambda x.\psi], & \sigma = \lambda x.\varphi\ \psi,\quad x \in FV(\varphi)\cup FV(\psi)
\end{array}\right.$$

Заметим, что данные равенства объясняют смысл названий комбинаторов:

\begin{tabular}{ll}
$S$ & verSchmelzung, <<сплавление>> \\
$K$ & Konstanz \\
$I$ & Identit\"at
\end{tabular}

\item Покажите, что следующая система комбинаторов образует полный базис в бестиповом
лямбда-исчислении, но соответствующая им система аксиом в исчислении гильбертового типа
не образует полного базиса для импликативного фрагмента:

$$\begin{array}{l}
I := \lambda x.x\\
K := \lambda x.\lambda y.x\\
S' := \lambda i.\lambda x.\lambda y.\lambda z.i\ (i\ ((x\ (i\ z))\ (i\ (y\ (i\ z)))))
\end{array}$$

Указание: покажите невыводимость $(\varphi \rightarrow \varphi \rightarrow \psi) \rightarrow (\varphi \rightarrow \psi)$.

\item Напомним определение аппликативного порядка редукции:
редуцируется самый левый из самых вложенных редексов. Например, в случае выражения
$(\lambda x.I\ I)\ (\lambda y.I\ I)$ самые вложенные редексы --- применения $I\ I$:

$$(\lambda x.\underline{I\ I})\ (\lambda y.\underline{I\ I})$$

и надо выбрать самый левый из них:

$$(\lambda x.\underline{I\ I})\ (\lambda y.I\ I)$$
\begin{enumerate}
\item Проведите аппликативную редукцию выражения $2\ 2$.
\item Докажите или опровергните, что параллельная бета-редукция из теоремы Чёрча-Россера не медленнее 
(в смысле количества операций для приведения выражения к нормальной форме), чем нормальный порядок 
редукции с мемоизацией.
\item Найдите лямбда-выражение, которое редуцируется медленнее при нормальном порядке редукции,
чем при аппликативном, даже при наличии мемоизации.
\end{enumerate}

\item Напомним определение бета-редукции. $A \rightarrow_\beta B$, если:
\begin{itemize}
\item $A \equiv (\lambda x.P)\ Q$, $B \equiv P\ [x := Q]$, при условии свободы для подстановки;
\item $A \equiv (P\ Q)$, $B \equiv (P'\ Q')$, при этом $P \rightarrow_\beta P'$ и $Q = Q'$, либо $P = P'$ и $Q \rightarrow_\beta Q'$;
\item $A \equiv (\lambda x.P)$, $B \equiv (\lambda x.P')$, и $P \rightarrow_\beta P'$.
\end{itemize}

\begin{enumerate}
\item Найдите лямбда-выражение, бета-редукция которого не может быть произведена из-за нарушения
правила свободы для подстановки (для продолжения редукции потребуется производить переименование
связанных переменных). Поясните, какое ожидаемое ценное свойство будет нарушено, если ограничение
правила проигнорировать.
\item Покажите, что недостаточно наложить требования на исходное выражение, и свобода для подстановки
может быть нарушена уже в процессе редукции исходно полностью корректного лямбда-выражения.
\end{enumerate}

\item Будем говорить, что выражение $A$ находится в \emph{слабой заголовочной нормальной форме} (WHNF),
если оно не имеет вид $A \equiv (\lambda x.P)\ Q$ (то есть, самый верхний терм его не является редексом).
Выражение находится в \emph{заголовочной нормальной форме} (HNF), когда его верхний терм --- не редекс и не лямбда-абстракция
с бета-редексами в теле.
\begin{enumerate}
\item Приведите нормальным порядком редукции выражение $2\ 2$ в СЗНФ.
\item Приведите нормальным порядком редукции выражение $Y\ (\lambda f.\lambda x.(IsZero\ x)\ 1\ (x \cdot f(x-1)))\ 3$ в СЗНФ.
\item Верно ли, что <<нормальность>> формы выражения может в процессе редукции только усиливаться
(никакая --- слабая заголовочная Н.Ф. --- заголовочная Н.Ф. --- нормальная форма)?
\end{enumerate}

\item Как мы уже разбирали, $\not\vdash x\ x:\tau$ в силу дополнительных ограничений
правила
$$\infer[x \notin FV(\Gamma)]{\Gamma, x:\tau\vdash x:\tau}{}$$

Найдите лямбда-выражение $N$, что $\not\vdash N:\tau$ в силу ограничений правила
$$\infer[x \notin FV(\Gamma)]{\Gamma \vdash \lambda x.N:\sigma\to\tau}{\Gamma, x:\sigma \vdash N:\tau}$$

%\item Верно ли, что $S = B(BW)(BBC)$? Если нет, то как правильно?

\end{enumerate}

\section*{Домашнее задание №2: задачи типизации лямбда исчисления}

\begin{enumerate}
\item Рассмотрим подробнее отличия исчисления по Чёрчу и по Карри. 
Определим точно бета-редукцию в исчислении по Чёрчу: $A \rightarrow_\beta B$, если

\begin{tabular}{ll}
$A = (\lambda x^\sigma.P)\ Q$, & $B = P [x := Q]$\\
$A = P\ Q$, & $B = P\ Q'$ или $B = P'\ Q$ при $P \rightarrow_\beta P'$ и $Q \rightarrow_\beta Q'$\\
$A = \lambda x^\sigma.P$, & $B = \lambda x^\sigma.P'$ при $P \rightarrow_\beta P'$
\end{tabular}
\begin{enumerate}
\item Покажите, что в исчислении по Карри не выполняется даже <<ограниченное>> свойство распространения типизации
(subject expansion): 
если $\vdash_\text{к} M:\sigma$, $M \twoheadrightarrow_\beta N$ и $\vdash_\text{к} N:\tau$,
то необязательно, что $\sigma=\tau$.
\item Покажите, что в исчислении по Чёрчу <<полное>> свойство распространения типизации также не выполняется:

\begin{center}найдутся такие $M,N,\sigma$, что $\vdash_\text{ч} N:\sigma$, $M\twoheadrightarrow_\beta N$, но $\not\vdash_\text{ч} M:\sigma$.\end{center}

Но при этом в исчислении по Чёрчу выполнено <<ограниченное>> свойство распространения типизации:

\begin{center}если $\vdash_\text{к} M:\sigma$, $M \twoheadrightarrow_\beta N$ и $\vdash_\text{к} N:\tau$,
то тогда $\sigma=\tau$.\end{center}
\end{enumerate}

\item Покажите, что никакие связки в ИИВ не выражаются друг через друга: то есть, нет такой формулы $\varphi(A,B)$ из языка 
интуиционистской логики, не использующей связку $\star$, что $\vdash A \star B \rightarrow \varphi(A,B)$ и $\vdash\varphi(A,B) \rightarrow A \star B$.
Покажите это для каждой связки в отдельности:
\begin{enumerate}
\item конъюнкция;
\item дизъюнкция;
\item импликация;
\item отрицание.
\end{enumerate}

\item Рассмотрим алгоритм построения системы уравнений, а именно случай, когда рассматривается два разных вхождения
одинакового по тексту применения. Например, $(x\ x)\ (x\ x)$ имеет два разных вхождения одной и той же аппликации
$x\ x$. Всегда ли для корректной работы алгоритма достаточно одной типовой переменной $\beta_{xx}$ для этих двух вхождений, 
или нужны две разные $\beta^L_{xx}$ и $\beta^R_{xx}$? 
Примечание: при одной переменной для обоих аппликаций система в данном случае, очевидно, несовместна: 
$\beta_{xx} \ne \beta_{xx} \rightarrow \sigma$.
Но контрпримером это не является, поскольку типа у данного выражения всё равно нет.

\item Предложим альтернативные аксиомы для конъюнкции:

$$\infer[\text{Введ. }\with]{\Gamma\vdash \alpha\with \beta}{\Gamma\vdash \alpha\ \ \ \Gamma\vdash \beta}\quad\quad
  \infer[\text{Удал. }\with]{\Gamma\vdash \gamma}{\Gamma\vdash \alpha\with \beta\ \ \ \Gamma, \alpha, \beta\vdash \gamma}$$

\begin{enumerate}
\item Предложите лямбда-выражения, соответствующие данным аксиомам; поясните, как данные выражения 
абстрагируют понятие <<упорядоченной пары>>.
\item Выразите изложенные в лекции аксиомы конъюнкции через приведённые в условии.
\item Выразите приведённые в условии аксиомы конъюнкции через изложенные в лекции.
\end{enumerate}

\item Постройте систему уравнений для $Y$-комбинатора и примените к ней алгоритм унификации (ожидается,
что система окажется несовместной).
\end{enumerate}

\section*{Домашнее задание №3: сильная нормализуемость $\lambda_\rightarrow$, система $F$}

\begin{enumerate}
\item Найдите $\llbracket\alpha\rightarrow\alpha\rrbracket$.
\item Найдите $\llbracket(\alpha\rightarrow\alpha)\rightarrow\alpha\rrbracket$.
\item Покажите, что SN --- насыщенное (постройте полноценное рассуждение по индукции для п.2 определения).
\item Покажите, что если $\mathcal{A}$ и $\mathcal{B}$ насыщены, то $\mathcal{A}\rightarrow\mathcal{B}$ насыщенное.

\item Покажите, что построенная на лекции простая модель для ИИП второго порядка неполна.
\item Напомним, что мы можем задать $\exists p.\varphi$ как $\forall q.(\forall p.\varphi\rightarrow q)\rightarrow q$
(где $q$ --- некоторая свежая переменная).
Покажите, что правила для квантора существования могут быть выведены из такого определения.
\item Требуется ли свобода для подстановки в правилах с квантором?

    \[ \dfrac{\Gamma\vdash\varphi[p := \theta]}{\Gamma\vdash\exists p.\varphi} \qquad
        \dfrac{\Gamma\vdash\forall p.\phi}{\Gamma\vdash\phi[p:=\theta]} \]

Если да --- приведите пример доказуемой при отсутствии свободы для подстановки, но некорректной формулы. 
Если нет --- предложите доказательство корректности правил при любых подстановках.

\item Пусть $\Gamma\vdash\varphi$. Всегда ли можно перестроить доказательство $\varphi$, добавив ещё одну гипотезу:
$\Gamma,\psi\vdash\varphi$? Если нет, каковы могли бы быть ограничения на $\psi$?

\item Пусть $\Gamma\vdash\forall x.\varphi$. Верно ли тогда, что $\Gamma\vdash\forall y.\varphi[x := y]$? 
Если это неверно в общем случае, возможно, это верно при каких-то ограничениях? В случае наличия ограничений
приведите надлежащие контрпримеры.

\item Перенесите в систему $F$ из бестипового лямбда-исчисления следующие функции --- иными словами,
постройте их обобщение для системы $F$ (приведите обобщённое выражение, укажите его тип и докажите его).
Например, можно рассмотреть $I = \Lambda \pi.\lambda p^\pi.p \rightarrow p$. 
\begin{enumerate}
\item S, K.
\item инъекции и $case$ (операции с алгебраическим типом);
\item истина, ложь, исключающее или;
\item черчёвский нумерал (он должен иметь тип $\forall\alpha.(\alpha\rightarrow\alpha)\rightarrow(\alpha\rightarrow\alpha)$) и инкремент;
\item возведение в степень: $\lambda m.\lambda n.n\ m$;
\item вычитание единицы (трюк зуба мудрости) и вычитание.
\end{enumerate}
\end{enumerate}

\section*{Домашнее задание №4: экзистенциальные типы, типовая система Хиндли-Милнера}
\begin{enumerate}
\item Постройте экзистенциальный тип для очереди, и реализуйте его с помощью двух стеков.
Реализацию напишите на Хаскеле, используя AbstractStack с лекции как АТД стека 
(возможно, этот пример надо будет расширить нужными вам методами), и реализуйте 
какой-нибудь простой классический алгоритм с её помощью. Как, интересно, осуществить
инстанциацию вложенного абстрактного типа данных? Придумайте.

Как с помощью двух стеков можно реализовать очередь со средним временем доступа $\Theta(1)$:
входные значения кладём во входной стек, выходные достаём из выходного, при исчерпании ---
переносим всё из входного в выходной:

\begin{tabular}{lll}
Входной стек & Выходной стек & Действие\\
$[]\rightarrow[1]$ & $[]$ & $push\ 1$\\
$[1]\rightarrow[3;1]$ & $[]$ & $push\ 3$\\
$[3;1]\rightarrow[]$ & $[]\rightarrow[1;3]\rightarrow[3]$ & $pull$\\
$[]\rightarrow[5]$ & $[3]$ & $push\ 5$
\end{tabular}

\item Выразите дизъюнкцию через квантор существования в ИИП 2 порядка, а также алгебраический тип через экзистенциальный.

\item Покажите, что если $rk(\sigma,1)$, то для выражения $\sigma$ найдётся эквивалентное $\sigma'$ с поверхностными
кванторами. 

\item Покажите, что в предыдущем задании также имеется изоморфизм типов: существует биективная функция $\sigma\rightarrow\sigma'$,
которую можно выразить лямбда-выражениями.

\item Рассмотрим QuickSort:
\begin{verbatim}
let rec quick l =  match l with
    [] -> []
  | l1 :: ls -> List.filter (fun x -> x < l1) ls @ [l1] @ List.filter (fun x -> x >= l1)
\end{verbatim}

Укажите полные типовые схемы в системе HM для всех функций, участвующих в данном примере (тип списка раскрывать не надо).

\item Заметим, что список --- это <<параметризованные>> числа в 
аксиоматике Пеано. Число --- это длина списка, а к каждому штриху мы присоединяем какое-то значение.
Операции добавления и удаления элемента из списка --- это операции прибавления и вычитания
единицы к числу.

Рассмотрим тип <<бинарного списка>>:

\begin{verbatim}
type 'a bin_list = Nil | Zero of (('a*'a) bin_list) | One of 'a * (('a*'a) bin_list);;
\end{verbatim}

Заметим, что здесь мы рассматриваем двоичную запись числа (чередующиеся \verb!Zero! и \verb!One!) ---
двоичную запись длины бинарного списка, и элемент двоичной записи номер $n$ хранит $2^n$ или $2^n+1$ 
значение (в зависимости от типа элемента). Например, 5-элементный список:

\begin{verbatim}
One ("a", Zero (One ((("b","c"),("d","e")), Nil)))
\end{verbatim}

По идее, операция добавления элемента к списку записывается на языке Окамль вот так 
(сравните с прибавлением 1 к числу в двоичной системе счисления):

\begin{verbatim}
let rec add elem lst = match lst with
    Nil -> One (elem,Nil)
  | Zero tl -> One (elem,tl)
  | One (hd,tl) -> Zero (add (elem,hd) tl)
\end{verbatim}

Однако, тип этой функции Окамль вывести автоматически не может, его надо указывать явно,
чтобы код скомпилировался:

\begin{verbatim}
let rec add : 'a . 'a -> 'a bin_list -> 'a bin_list = fun elem lst -> match lst with
\end{verbatim}

\begin{enumerate}
\item Какой тип имеет \verb!add! в (расширенной) системе $F$ (напомним, поскольку функция рекурсивна,
она должна использовать Y-комбинатор в своём определении)?
Считайте, что семейство типов \verb!bin_list 'a! предопределено и обозначается как $\tau_\alpha$.
Также считайте, что определены функции roll и unroll с надлежащими типами.
Какой ранг имеет тип этой функции? Почему этот тип не выразим в типовой системе Хиндли-Милнера?
\item Предложите функции для печати списка и для удаления элемента списка (головы).
\item Предложите функцию для эффективного соединения двух списков (источник для 
вдохновения --- сложение двух чисел в столбик).
\item Предложите функцию для эффективного выделения $n$-го элемента из списка.
\end{enumerate}

\item Рассмотрим следующий код на Окамле, содержащий определения чёрчевских нумералов
и некоторых простых операций с ними:

\begin{verbatim}
let zero = fun f x -> x;;
let plus1 a = fun f -> fun x -> a f (f x);;
let power m n = n m;;

let two = plus1 (plus1 zero);;
let two2 = fun f x -> f (f x);;

let e  = power two two;;          (* не компилируется *)
let e2 = power two2 two2;;        (* компилируется и работает *)
\end{verbatim}

Разберите вывод типов в этом фрагменте (относительно типовой системы Хиндли-Милнера) и поясните, почему:
\begin{enumerate}
\item определение $e2$ компилируется и работает (предъявите доказательство типа в системе HM);
\item определение $e$ не компилируется (например, примените алгоритм W и покажите шаг, где он выведет ошибку).
\end{enumerate}

\end{enumerate}


\section*{Домашнее задание №5: Обобщённая система типов, гомотопическая теория типов, язык Аренд}

\begin{enumerate}
\item Задача на доказательство сильной нормализуемости $\lambda_\rightarrow$: найдите примеры лямбда-термов,
не принадлежащие (1) множеству $\llbracket \alpha \rightarrow \alpha\rrbracket$ и (2) множеству $\llbracket (\alpha \rightarrow \alpha) \rightarrow \alpha \rrbracket$
(для выполнения задания надо выполнить оба пункта).
\item Укажите тип (род) в исчислении конструкций для следующих выражений (при необходимости определите
типы используемых базовых операций и конструкций самостоятельно):
\begin{enumerate}
\item В алгебраическом типе \verb!'a option = None | Some 'a! предложите тип (род) для: \verb!Some!,
\verb!None! и \verb!option!.
\item Пусть задан род $\textbf{nonzero}: \star\rightarrow\star$, выбрасывающий нулевой элемент из
типа. Например, $\textbf{nonzero}\ \textbf{unsigned}$ --- тип положительных целых чисел.
Тогда, для кода
\begin{verbatim}
template<typename T, T x>
struct NonZero { const static std::enable_if_t<x != T(0), T> value = x; };
\end{verbatim}
предложите тип (род) поля value.
\end{enumerate}

\item Предложите выражение на языке C++ (возможно, использующее шаблоны), имеющее следующий род (тип):
\begin{enumerate}
\item $\star\rightarrow\star\rightarrow\star$; $\ \star\rightarrow\textbf{unsigned}$
\item $\textbf{int}\rightarrow(\star\rightarrow\star)$
\item $(\star\rightarrow\textbf{int})\rightarrow\star$
\item $\Pi x^\star.\lambda n^\textbf{int}.F(n,x)$, где $$F(n,x) = \left\{\begin{array}{ll}\textbf{int}, & n = 0\\
                                   x\rightarrow F(n-1,x), & n > 0\end{array}\right.$$
\end{enumerate}

\item Определите функции из следующих частей $\lambda$-куба (в обобщённой типовой системе) и докажите их тип:
\begin{enumerate}
\item $(\square,\star)$
\item $(\star,\square)$
\item $(\square,\square)$
\end{enumerate}

\item Рассмотрим правый дальний нижний угол $\lambda$-куба ($\{(\star,\star);(\star,\square);(\square,\star)\}$).
Можно предположить, что тогда в такой системе возможны и функции рода $f: \star\rightarrow\star$ 
(как композиция функций $p: \star\rightarrow\alpha$ и $q: \alpha\rightarrow\star$ ---
например, можно кодировать тип его именем, затем по имени типа восстанавливать сам тип обратно).
Почему всё-таки такие функции в обобщённых типовых системах невозможны без четвёртого элемента $(\square,\square)$?

\item Какова должна быть топология на множестве
пар натуральных чисел (интуитивно мы будем понимать эти пары как рациональные числа, пары <<числитель-знаменатель>>), 
чтобы непрерывными были бы те и только те функции, для которых выполнено $f(p,q) = f(p',q')$ для всех 
таких $p,p',q$ и $q'$, что $p\cdot q' = p'\cdot q$. Напомним, что равенство мы понимаем как наличие непрерывного пути 
между точками.

\item Докажите, приведя компилирующуюся программу на языке Аренд (возможно, вам потребуются функции и приёмы, 
изложенные в документации по языку: \url{https://arend-lang.github.io/documentation/tutorial/PartI/}):
\begin{enumerate}
\item ассоциативность сложения;
\item коммутативность сложения;
\item коммутативность умножения;
\item дистрибутивность: $(a + b)\cdot c = a\cdot c + b \cdot c$;
\item куб суммы: $(a + 1)^3 = a^3 + 3\cdot a^2 + 3 \cdot a + 1$.
\end{enumerate}

\item Определим, что $x$ делится на $p$, если обитаем тип \verb!\Sigma (q : Nat) (p * q = x)!.
\begin{enumerate}
\item Покажите, что если $x$ делится на 6, то $x$ делится и на 3;
\item Покажите, что $x!$ делится на $x$;
\item Покажите, что если $x$ делится на $y$ и $y$ делится на $z$, то $x$ делится на $z$;
\end{enumerate}

\item Определите предикат (т.е. функцию с надлежащим типом) для формализации понятия простого числа \verb!isPrime!.
Покажите, что:
\begin{enumerate}
\item 3 и 11 --- простые числа;
\item Произведение простых чисел непросто;
\item 2 --- единственное чётное простое число.
\end{enumerate}

\item Определим отношение <<меньше>> на натуральных числах так (с помощью индуктивного типа, обобщения алгебраического типа данных ---
зависимого типа, в котором при разных значениях аргументов типа допустимы разные конструкторы):
\begin{verbatim}
\data NatLessEq (a b : Nat) \with
  | 0, m => natlesseq-zero
  | suc m, suc n => natlesseq-next (NatLessEq m n)
\end{verbatim}

Например, конструктор \verb!natlesseq-zero! можно использовать только если первый аргумент типа --- число 0.
А конструктор \verb!natlesseq-next! применим только если первый аргумент больше 0; при этом данный конструктор
требует значение типа с определёнными аргументами в качестве своего аргумента.

Будем говорить, что $a \preceq b$ тогда и только тогда, когда \verb!NatLessEq a b! обитаем.
Например, утверждение $1 \preceq 3$ доказывается так:

\begin{center}\verb!\func one-le-three : NatLessEq 1 3 => natlesseq-next (natlesseq-zero)!\end{center}

\emph{В самом деле}, \verb!natlesseq-zero! может являться конструктором типа \verb!NatLessEq 0 b!,
а тогда \begin{center}\verb!natlesseq-next (natlesseq-zero) : NatLessEq 1 (b+1)!\end{center} Унифицировать $b+1$ и $3$
компилятор (в данном случае) может самостоятельно, и потому код выше проходит проверку на корректность.

Докажите (везде предполагается, что $a,b,c : \texttt{Nat}$, если не указано иного):
\begin{enumerate}
\item $a \preceq b$ тогда и только тогда, когда $a$ меньше или равно $b$ в смысле натуральных чисел (здесь требуется рассуждение
на мета-языке).
\item $a \preceq a + b + 1$; то есть, определите функцию\\\verb!\func n-less-sum (a b : Nat) : NatLessEq a (a Nat.+ suc b)!
\item Если $a \preceq b$, то $a + c \preceq b + c$
\item Если $a \preceq b$ и $c \preceq d$, то $a \cdot c \preceq b \cdot d$
\item $a \preceq 2^a$
\item Транзитивность: если $a \preceq b$ и $b \preceq c$, то $a \preceq c$
\item $a \preceq b \vee b \preceq a$
\item Найдите стандартное определение отношения <<меньше>> в библиотеке Аренда (\verb!Nat.<!) и докажите, что $a \preceq b$ 
тогда и только тогда, когда $a < b$ или $a = b$ (реализуйте функции \\\verb!there (p : NatLessEq a b) : Data.Or (a Nat.< b) (a = b)!
и обратную к ней).
\item Покажите, что $a \preceq b$ тогда и только тогда, когда $\exists k^{\mathbb{N}_0}.a + k = b$.
\end{enumerate}

\item С точки зрения изоморфизма Карри-Ховарда индуктивные типы можно воспринимать как аналоги предикатов.
В этом задании надо построить индуктивные типы для различных предикатов:
\begin{enumerate}
\item Факториал (\verb!IsFact n!), который будет обитаем только для таких $n$, что $n = k!$.
Докажите на языке Аренд, что \verb!IsFact! $(1 \cdot 2 \cdot 3 \cdot \dots \cdot n)$ всегда обитаем, а
тип \verb!IsFact 3! --- необитаем.
\item Наибольший общий делитель двух чисел \verb!GCD x a b!; \emph{указание/пожелание:} воспользуйтесь алгоритмом Эвклида.  
\item Ограниченное натуральное число \verb!IndFin n!, обитателями типа являются только те числа, которые меньше $n$.
В стандартной библиотеке \verb!Fin! определён через натуральные числа; 
сделайте это исключительно через индуктивные типы. Покажите, что если \verb!x : IndFin m! и \verb!y : IndFin n!,
то \verb!x + y : IndFin (m + n)!.
\end{enumerate}
\end{enumerate}

\section*{Домашнее задание №6: Иерархии универсумов}
\begin{enumerate}

\item Рассмотрим следующее доказательство уникальности элементов списка:
\begin{verbatim}
\func not-member (A : \Type) (elem : A) (l : List A) : \Type \elim l
  | nil => \Sigma
  | :: hd tl => \Sigma (Not (hd = elem)) (not-member A elem tl)

\func unique-list (A : \Type) (l : List A) : \Type \elim l
  | nil => \Sigma
  | :: hd tl => \Sigma (not-member A hd tl) (unique-list A tl)
\end{verbatim}

Докажем, что список $[0,1,2]$ состоит из уникальных элементов:
\begin{verbatim}
\func r-unique => unique-list Nat (0 :: 1 :: 2 :: nil)
\func x : r-unique => ((contradiction, (contradiction, ())), ((contradiction, ()), ((), ())))
\end{verbatim}

\begin{enumerate} 
\item Напишите функцию, строящую список \verb!b! натуральных чисел от 0 до n и доказательство\\ \verb!unique-list Nat b!.
\item Покажите, что если $a_0 < a_1 < \dots < a_n$, то список $[a_0,a_1,\dots,a_n]$ уникальный.
\item Покажите, что если $n \ge 2$ и $a_k \ne a_{k+1}$ при $0 \le k < n$, то необязательно, что список
$[a_0,a_1,\dots,a_n]$ --- уникальный.
\end{enumerate}

\item Определите тип \verb!Perm n! --- его элементами должны быть те и только те списки чисел, которые являются перестановкой
$n$ элементов --- и покажите, что:
\begin{enumerate}
\item $0,1,2,\dots,n-1$ --- перестановка $n$ элементов;
\item определите, чему равна сумма всех элементов перестановки --- и докажите что это действительно так для любого $n$;
\item всего существует $6$ элементов в типе \verb!Perm 3! (то есть, существует такой список из 6 элементов,
каждые два элемента которого не равны друг другу, и если \verb!x : Perm 3!, то $x$ --- элемент данного списка).
\end{enumerate}

\item Как вы помните из лекции, в языке Аренд существует иерархия вложенных типовых универсумов. Если в типе отсутствует
упоминание \verb!\Type!, то данный тип принадлежит универсуму 0. Однако, если в типе упоминается \verb!\Type k!,
то тип принадлежит универсумам, не меньшим $k+1$. Уровень универума обозначается специальным ключевым словом \verb!\lp!.
Над индексами можно проводить простые операции и сопоставление с образцом (\verb!\suc\lp!). Более подробно
можно это прочесть в документации по языку Аренд:

\url{https://arend-lang.github.io/documentation/tutorial/PartI/universes.html}
%\begin{enumerate}
%\item Определите функцию \verb!\id!, возвращающую 
%\end{enumerate}

Рассмотрим определения:
\begin{verbatim}
\func ChurchT (x : \Type) => (x -> x) -> (x -> x)
\func Church => \Pi (x : \Type) -> ChurchT x
\func Zero : Church => \lam t f x => x

\func incT (t : \Type) (n : ChurchT t) => \lam f x => n f (f x)
\func pair_plus (type : \Type) (pair : \Sigma (ChurchT type) (ChurchT type)) :
  \Sigma (ChurchT type) (ChurchT type) => (pair.2, incT type pair.2)
\func dec (n : Church) : Church => \lam (t : \Type) =>
    (n (\Sigma (ChurchT t) (ChurchT t)) (pair_plus t) (Zero t, Zero t)).1
\func sub (a : Church (\suc\lp)) (b : Church) => a (Church \lp) dec b
\end{verbatim}

Определите, развивая определения выше:
\begin{enumerate}
\item Операцию умножения.
\item Операцию <<деление на три>> (естественно, в версии, не использующей $Y$-комбинатор).
\item Операцию возведения в степень, определявшуюся как $\lambda m.\lambda n.n\ m$.
\item Деление.
\item Вычисление факториала.
\end{enumerate}


\item Введём тип данных \verb!IsEven!:

\begin{verbatim}
\data IsEven (n : Nat) \elim n
  | 0 => zero-is-even
  | (suc (suc k)) => next-next (IsEven k)
\end{verbatim}

Доказать, что этот тип является утверждением, можно например так:

\begin{verbatim}
\func all-even-different (n : Nat) (a b : IsEven n) : a = b \elim n, a, b
  | 0, zero-is-even, zero-is-even => idp
  | suc (suc n), next-next a, next-next b => pmap next-next (all-even-different n a b)

\func is-even-isProp (n : Nat) : isProp (IsEven n) => all-even-different n
\end{verbatim}

Обратите внимание: по переменным $n$, $a$ и $b$ производится \emph{элиминация}, то есть множество 
значений переменных разбивается на фрагменты (в соответствии с конструкторами типа данных), 
и доказательство утверждения проводится независимо для каждого фрагмента; 
в частности, цель доказательства изменяется и просходит замена переменных $a$ и $b$ на сопоставляемые
варианты (вместо ожидаемого типа \verb!a = b! мы ожидаем тип \verb!zero-is-even = zero-is-even!, и т.п.).
Сравните с лекцией про элиминаторы, каждый из вариантов --- тело отдельной функции-элиминатора. 

Чтобы воспроизвести тот же эффект в конструкции \verb!\case!, нужно указывать ключевое слово 
\verb!\elim! перед каждой элиминируемой переменной: 
\begin{verbatim}
\case \elim n, \elim a, \elim b \with { ... }
\end{verbatim}

Полный код, определяющий тип \verb!IsEven! (вместе с доказательством того, что тип --- утверждение),
выглядит так:

\begin{verbatim}
\data IsEven (n : Nat) \elim n
  | 0 => zero-is-even
  | (suc (suc k)) => next-next (IsEven k)
  \where {
    \func all-even-different ... -- скопируйте код функции сюда
    \use \level is-even-isProp (n : Nat) : isProp (IsEven n) => all-even-different n
  }
\end{verbatim}

Однако, незавершённым остаётся доказательство разрешимости типа \verb!IsEven n!. Восполните лакуны:

\begin{verbatim}
\func even-is-dec (a : Nat) : Dec (IsEven a) \elim a
  | 0 => yes zero-is-even
  | 1 => no {?}
  | suc (suc a) => {?}
\end{verbatim}

\item Справедливости ради, в предыдущем задании компилятор сам может догадаться, что \verb!IsEven! --- утверждение.
Но далеко не для всех типов это очевидно, и тогда функция с префиксом \verb!\use \level! становится необходимой.
Например, в следующем типе <<простое число>> данная функция должна доказать, что любые два значения типа при данном $x$ равны:

\begin{verbatim}
\data Div3 (x : Nat)
  | remainder-zero (Exists (p : Nat) (p Nat.* 3 = x))
  | remainder-one (Exists (p : Nat) (p Nat.* 3 Nat.+ 1 = x))
  | remainder-two (Exists (p : Nat) (p Nat.* 3 Nat.+ 2 = x))
  \where \use \level levelProp {x : Nat} (a b : Div3 x) : a = b => {?}
\end{verbatim}

\begin{enumerate}
\item Замените \verb!{?}! в тексте выше на корректное доказательство.
\item Определите функцию, которая бы по $x$ и значению $\exists p q.3 \cdot p + q = x \with 0 \le q < 3$ возвращала бы \verb!Div3 x!
(понятно, можно разделить $x$ на 3, но нам уже результат деления дали вторым аргументом --- задача в том, чтобы им воспользоваться).
\item Постройте аналогичный тип \verb!Prime x! для простых чисел --- с тремя вариантами \verb!less-than-two!, \verb!is-prime!, 
\verb!is-composite! --- и определите функцию, строящую по числу значение данного типа.
\item Покажите, что в типе \verb!Prime (x*x + 2*x + 1)! всегда (кроме граничных случаев) обитает вариант \verb!is-composite!.
\item Покажите, что в типе 
\begin{verbatim}
\data SuperDec (P : \Prop)
| sure P
| nope (P -> Empty)
| neither ((P || (P -> Empty)) -> Empty)
\where \use \level superDecProp {P : \Prop} (a b : SuperDec P): a = b => {?}
\end{verbatim}
вариант \verb!neither! невозможен (также, заполните пропущенное доказательство \verb!superDecProp!).
\end{enumerate}

\item Рассмотрим определение целых чисел как упорядоченной пары: $\langle a,b\rangle \subseteq \mathbb{N}^2$,
причём $\langle a,b\rangle \approx \langle c,d\rangle$, если $a + d = b + c$. Построим соответствующий тип
данных, $\mathbb{N}^2/\approx$:

\begin{verbatim}
\data Integer
  | int_data (l r : Nat)
  | int_eq (a b c d : Nat) (a Nat.+ d = b Nat.+ c) : int_data a b = int_data c d
\end{verbatim}

Обратите внимание на второй конструктор \verb!int_eq! --- это специальный конструктор для отношения эквивалентности,
он постулирует равенство между элементами, и его можно использовать для доказательства такого равенства.
Указание на особую роль конструктора --- указание его типа, и значением конструктора являются не сами элементы типа
\verb!Integer! (как это имеет место в случае \verb!int_data!), а пути между элементами типа \verb!Integer!.
Покажем, например, что $[\langle 1,3\rangle] = [\langle 0,2\rangle]$:

\begin{verbatim}
\func r : int_data 1 3 = int_data 0 2 => int_eq 1 3 0 2 idp
\end{verbatim}

Теперь мы можем определить функции на целых числах:

\begin{verbatim}
\func inc (a : Integer) : Integer \elim a
  | int_data l r => int_data (suc l) r
  | int_eq a b c d proof => int_eq (suc a) b (suc c) d (pmap suc proof)
\end{verbatim}

Обратите внимание, мы должны указать не только образы для всех элементов \verb!Integer! (первый случай), но и показать, что 
равные элементы перешли в равные (второй случай).
А именно, нам потребовалось доказать, что операция прибавления 1 вернёт эквивалентные числа для эквивалентных аргументов:
если $\langle a,b\rangle \approx \langle c,d\rangle$, то $\langle a+1, b\rangle \approx \langle c+1,d\rangle$.

\begin{enumerate}
\item Определите функцию \verb!isPositive : Integer -> \Type!, результат которой обитаем тогда и только тогда,
когда аргумент --- положительное число.
\item Покажите \verb!Dec (isPositive k)! для любого целого $k$.
\item Определите функцию \verb!dec : Integer -> Integer!, уменьшающую число на 1.
\item Докажите, что \verb!inc (dec x) = x!.
\item Определите функцию \verb!abs : Integer -> Nat!, возвращающую модуль целого числа.
\item Определите функцию \verb!plus : Integer -> Integer -> Integer!, складывающую два числа.
\end{enumerate}

\item Заметим, что получившийся тип \verb!Integer! множеством (\verb!\Set!) не будет. Чтобы такое выполнялось, необходимо
показать равенство равенств элементов ($\Pi x^\texttt{Integer}.\Pi y^\texttt{Integer}.\Pi a^{x = y}.\Pi b^{x = y}.a = b$). 
Это можно сделать (точнее, \emph{постулировать}), например, с помощью конструкции \verb!\truncate!:

\begin{verbatim}
\truncate \data Integer : \Set
  | int_data (l r : Nat)
  | int_eq (a b c d : Nat) (a Nat.+ d = b Nat.+ c) : int_data a b = int_data c d
\end{verbatim}

И далее, чтобы показать равенство равенств, мы сможем воспользоваться библиотечной функцией \verb!Path.inProp!
(без указания \verb!\truncate! данный код не скомпилируется):

\begin{verbatim}
\func integer_eq (x y : Integer) (p1 p2 : x = y) : p1 = p2 => Path.inProp p1 p2
\end{verbatim}

\begin{enumerate}
\item Поясните (на метаязыке, с точки зрения топологии), почему неусечённый тип \verb!Integer! --- не \verb!\Set!. 
%\emph{Указание:} 
%заметим, что путь $\langle 0, 0\rangle = \langle 0, 0 \rangle$ можно получить композицией путей $\langle 0, 0 \rangle = \langle 1,1 \rangle$
%и $\langle 1, 1\rangle = \langle 0, 0 \rangle$.
\item Определите функцию \verb!isSet: \Type -> \Type!, результат которой обитаем если исходный тип является множеством ---
в качестве источника вдохновения можно взять \verb!isProp!.
Докажите, что \verb!Nat! --- множество.
\item Использовать конструкцию \verb!\truncate! необязательно --- вместо неё можно указать надлежащий дополнительный конструктор
равенства (теперь для путей) --- и указать надлежащую функцию \verb!\use \level!:
\begin{verbatim}
 | eq_eq (a b : Integer) (p1 p2 : a = b) : p1 = p2
  \where { \use \level asSet ... }
\end{verbatim}

Дополните описание типа данных так, чтобы тип данных \verb!Integer! оказался множеством без специальной конструкции 
\verb!\truncate! (убедитесь, что Path.inProp p1 p2 теперь определён для путей на \verb!Ingeger!, 
и поэтому аналог \verb!integer_eq! скомпилируется).
Также исправьте определение функции \verb!inc! из прошлого задания.

\end{enumerate}

%\item Рассмотрим аксиому для квантора существования из исчисления предикатов (в гильбертовской форме):
%\begin{verbatim}
%\func exists-axiom {T : \Type} {A : T -> \Prop} (evidence : T) (proof : A evidence) 
%    : Exists (x : T) (A x) => inP (evidence, proof)
%\end{verbatim}
%
%Сформулируйте и докажите:
%\begin{enumerate}
%\item правило вывода из исчисления предикатов;
%\item правило удаления квантора существования из исчисления предикатов в натуральном выводе:
%$$\infer{\vdash\psi}{\vdash\exists x.\varphi(x)\quad\quad \varphi(\theta)\vdash\psi}$$
%\end{enumerate}

\item Научимся раскрывать усечённый тип данных (при возможности это сделать):
\begin{enumerate}
\item По $\exists p.p' = x$ постройте такой $p$, что $p' = x$: 
\begin{verbatim}\func safe-dec (x : Nat) (Exists (p : Nat) (suc p = x)) : \Sigma (p : Nat) (suc p = x)\end{verbatim}
\emph{Указание: } второй параметр нужен для того, чтобы исключить вариант $x = 0$.
\item По \verb!not-equals : (x > y || x < y)! при \verb!x y : Nat! 
(обратите внимание, здесь применяется усечённое <<или>>) получите \verb!x /= y!.
\item По \verb!x : Nat! и \verb!p : Exists (p : Nat) (p * p = x)! найдите
\verb!\Sigma (p : Nat) (p * p = x)! (мы здесь должны существенно использовать единственность натурального корня числа).
\end{enumerate}

\item В предыдущих заданиях мы строили фактор-множества вручную. То же можно сделать с помощью библиотечного типа данных \verb!Quotient!.
\begin{enumerate}
\item Постройте тип множества рациональных положительных чисел \verb!Rational! как фактор-множество пар $\langle a,b \rangle$ и $\langle c,d \rangle$
по отношению <<равны как простые дроби>>:
$\langle a, b \rangle \approx \langle c, d \rangle$, если $a\cdot d = b \cdot c$ ($a,c \in \mathbb{N}_0$, $b,d \in \mathbb{N}$).
\item Определите арифметические операции (сложение, умножение).
\item Постройте функцию \verb!to_rat (arg: Nat) : Rational! и \verb!from_rat! (выполняющую округление вниз). Покажите, что 
\verb!\Pi (x : Nat) -> x = from_rat (to_rat x)!.
\end{enumerate}
\end{enumerate}

\section*{Домашнее задание №7: Алнебраическая топология}

\begin{enumerate}
\item Докажите, что $\mathbb{R}^2$ и $\mathbb{R}\times [0,+\infty)$ гомотопически эквивалентны, но не гомеоморфны.
\item Докажите, что буквы \verb!O! и \verb!Q! гомотопически эквивалентны, но не гомеоморфны.
\item Покажите, что для любых двух мощностей $\alpha$ и $\beta$ найдутся два гомотопически эквивалентных пространства $A, B$: $|A|=\alpha$,
$|B|=\beta$.
\item Покажите, что пространство $X$ линейно связно тогда и только тогда, когда любые отображения $\{0\} \rightarrow X$ гомотопны.
\item Подсчитайте $\pi_1(S^2)$.
\item Рассмотрим деревья с топологией <<открыты множества, содержащие вершины со всеми своими потомками>>. Поясните, какие деревья
будут в такой топологии гомотопически эквивалентны.
\item Докажите, что \verb!(Bool = Bool) = Bool!.
\end{enumerate}

% написать absurd
\section*{Домашнее задание №8: Аксиома выбора, теорема Диаконеску}
\begin{enumerate}
\item Покажите (на метаязыке), что в ИИВ из закона исключённого третьего следует разрешимость (разумеется, не пользуясь разрешимостью ИИВ).
А также поясните, что из разрешимости следует закон исключённого третьего.
\item Сформулируйте на языке Аренд аксиому выбора для \verb!\Set!-ов. Покажите, что она является теоремой.
\item Покажите, что равенство для \verb!\Set!-ов разрешимо.
\item Определите сетоиды в Аренде. Покажите, что целые числа образуют сетоид.
\item Докажите теорему Диаконеску на Аренде с помощью сетоидов.
\item Поясните (на метаязыке) утверждение: <<аксиома выбора --- это аксиома о перестановке кванторов>>.
\item С помощью аксиомы выбора на Аренде покажите, что любая сюрьективная функция из факторизованного множества (файл стандартной
библиотеки \verb!Relation/Equivalence.ard!, тип данных \verb!Quotient!) в факторизованное имеет частичную обратную.
\item Покажите, что если любая сюрьективная функция из факторизованного множества (файл стандартной
библиотеки \verb!Relation/Equivalence.ard!, тип данных \verb!Quotient!) в факторизованное имеет частичную обратную, то выполнена аксиома выбора.
\end{enumerate}

\section*{Домашнее задание №9: парадокс Жирара}
\begin{enumerate}
\item Напомним, что для ординала $\alpha$ за $\sigma\alpha$ мы обозначили множество меньших ординалов,
а за $\tau X$ --- порядковый тип упорядоченного множества $\langle X,< \rangle$. 
Покажите, что для любого ординала $\alpha$ выполнено $\tau\sigma\alpha = \alpha$.
Верно ли, что $\alpha=\sigma\alpha$?
\item Пусть $f: S \rightarrow T$. Определим $f_\star : \wp S \rightarrow \wp T$ поэлементно: $f_\star(X) = \{ f(x)\ |\ x \in X \}$.
\begin{enumerate}
\item Покажите, что универсум $\mathcal{U}$ парадоксален тогда и только тогда, когда $\langle \wp\mathcal{U}, \sigma_\star, \tau_\star\rangle$ 
парадоксален. 
\item Рассмотрим $\mathcal{U} = \{\langle A, \prec, a \rangle\ |\ (\prec) \subseteq A^2, a \in A\}$.
Пусть $\sigma\langle A, \prec, a \rangle = \{\langle A, \prec, b\rangle\ |\ b \prec a\}$. Тогда $\sigma : \mathcal{U} \rightarrow \wp\mathcal{U}$.

Тогда мы можем ввести отношение на $\wp\mathcal{U}$. Будем считать $Y < X$, если $Y \in \sigma_\star X$.

Определим $\tau X := \langle \wp \mathcal{U}, (<), X\rangle$.

Покажите, что $\langle \mathcal{U}, \sigma, \tau\rangle$ --- парадоксальный универсум.
\end{enumerate}
\item Покажите, что $\Omega\in\mathcal{U}$.
\item Покажите невозможность фундированности $\Omega$. А именно, обоснуйте переходы и восполните пробелы в следующих рассуждениях:
\begin{enumerate}
\item Лемма. Множество $\{y | \neg\tau\sigma y < y \}$ индуктивно:
\begin{itemize}
\item Рассмотрим $x$, что для всех $y < x$ выполнено $\neg \tau\sigma y < y$.
\item Предположим, что $\tau\sigma x < x$. 
\item Тогда $\neg \tau\sigma\tau\sigma x < \tau\sigma x$.
\item Но $\tau\sigma\tau\sigma x = \tau\sigma u$ для некоторого $u < x$.
\end{itemize}
\item Основная теорема: $\Omega$ не фундировано.
\begin{itemize}
\item Предположим, что $\Omega$ фундировано.
\item $\tau\sigma\Omega$ имеет вид $\tau\sigma\omega$ для некоторого фундированного $\omega$.
\item $\tau\sigma\Omega$ есть предшественник $\Omega$.
\item Однако, $\neg \tau\sigma \Omega < \Omega$, поскольку $\Omega$ фундировано и $\{y | \neg\tau\sigma y < y \}$ индуктивно.
\end{itemize}
\end{enumerate}
\item Вспомним определение с лекции:
\begin{itemize}
\item $\mathcal{U} = \Pi X : \Box . ((\wp X \rightarrow X) \rightarrow \wp X)$
\item $\tau = \lambda X : \wp\mathcal{U}. \lambda A : \Box.\lambda c : (\wp A \rightarrow A).\lambda a : A.\varphi$, где
 $\varphi = \Pi P^{\wp A}.(\Pi x^\mathcal{U}.X\ x\rightarrow (P (c ((x\ A)\ c)))) \rightarrow P\ a$

\item $\sigma = \lambda x : \mathcal{U} . ((x\ \mathcal{U})\ \tau)$
\end{itemize}

\begin{enumerate}
\item Найдите типы $\tau$ и $\sigma$.
\item Покажите, что универсум $(\mathcal{U},\sigma,\tau)$ парадоксален, показав бета-эквивалентность:
для всех $X: \wp \mathcal{U}$ выполнено $\sigma\tau X =_\beta \bigcap \{ P \subseteq \mathcal{U}\ |\ \forall x \in X.\tau\sigma x \in P \}$.
В лямбда-исчислении нет примитива для аксиомы выделения, поэтому записать прямо формулу вида
$T = \{ \tau\sigma x\ |\ x \in X\}$ не получится --- но тот же результат можно получить, построив пересечение всех подмножеств $\mathcal{U}$, содержащих $T$
(или, говоря ближе к языку лямбда-исчисления, построив конъюнкцию всех предикатов $P$ со свойством $X\ x\rightarrow P(\tau\sigma x)$).
\end{enumerate}
\end{enumerate}

\section*{Домашнее задание №10: линейная логика, уникальные типы}
\begin{enumerate}
\item Как известно, интуиционистские связки могут быть выражены в линейном исчислении.
Покажите соответствующие интуиционистские аксиомы для следующих способов выразить интуиционистские связки
через линейные:
\begin{enumerate}
\item Интуиционистская дизъюнкция: $A + B :=\ !A \oplus !B$
\item Интуиционистская конъюнкция: $A \times B := A \with B$
\item Альтернативная конъюнкция: $A \times B :=\ !A \otimes !B$ 
\item Покажите, что альтернативная конъюнкция влечёт обычную, но не наоборот: $\langle!A \otimes !B\rangle \vdash A \with B$,
но $\langle A \with B\rangle \not\vdash \ !A \otimes !B$.
\end{enumerate}

\item Покажите следующие линейные тождества:
\begin{enumerate}
\item $\vdash \phi \multimap \phi$
\item $\vdash \ !\phi \multimap \ !!\phi$
\item $!(\alpha\&\beta) \equiv !\alpha\otimes!\beta$ (надо построить два доказательства, $\varphi\vdash\psi$ и $\psi\vdash\varphi$)
\end{enumerate}

\item Дистрибутивность
\begin{enumerate}
\item Выполняется ли дистрибутивность для $(\&)$ и $(\oplus)$?
\item Выполняется ли дистрибутивность для $(\&)$ и $(\otimes)$?
\end{enumerate}

\item В языке Раст есть тип функции, которую можно вызвать только один раз: \verb!FnOnce!.
Утверждается, что функция \verb!const x y = x! в типовой системе уникальных типов (как во второй части лекции) 
будет иметь тип $t^u \rightarrow^\times s^v \rightarrow^u t^u$ (то есть, вернёт
однократную функцию, если её первый аргумент уникальный). Покажите это, пользуясь правилами вывода
с лекции. Каков будет тип этой функции в языке Раст (вопрос на дополнительный балл)?

\item Рассмотрим $C_0 = [0,1]$ и удалим из него треть: $C_1 = [0,\frac{1}{3}] \cup [\frac{1}{3},1]$.
Повторим процесс --- $C_2 = [0,\frac{1}{9}] \cup [\frac{2}{9},\frac{3}{9}] \cup [\frac{6}{9},\frac{7}{9}] \cup [\frac{8}{9},1]$.
И так далее, для построения $C_{k+1}$ будем удалять по трети из каждого интервала, имеющегося в $C_k$.
Тогда $\bigcap C_i$ называется канторовым множеством.

Рассмотрим $\langle X_i, \Omega_i \rangle$ --- семейство топологических пространств.
Пусть $A \in \times \{\Omega_i\}$ --- семейство открытых множеств, по открытому множеству из каждого пространства.
Пусть $T(A)$ есть свойство, что $A$ совпадает с исходными пространствами почти везде: 
$$T(A) := \exists a_1,\dots,a_k \in \mathbb{N}.A_i = X_i\text{, если }i \notin \{a_1,\dots,a_k\}$$

Тогда: коробочная топология --- топология на $\times\{X_i\}$ с базой $\{ A | A \in \times \{\Omega_i\}\} $ (всевозможные произведения 
открытых множеств из семейства).

Тихоновская топология --- минимальная топология на произведении, что все проекции открытых множеств на компоненты произведения открыты
(это --- стандартная топология произведения).
Или иначе её можно задать как топологию с базой из всевозможных произведений открытых множеств,
причём только конечное количество множеств не равно всей компоненте.
То есть, топология с базой $\{A | A \in \times\{\Omega_i\}, T(A)\}$.

Очевидно, что в случае конечного произведения пространств коробочная и тихоновская топологии совпадают.

\begin{enumerate}
\item Покажите, что тихоновоская топология на $\{0,1\}^{\aleph_0}$ гомеоморфна канторову множеству (с топологией, индуцированной
евклидовой топологией). На $\{0,1\}$ топология дискретна. \emph{Указание:} Отрезку $[\frac{2}{9},\frac{3}{9}]$ по построению можно 
сопоставить двоичный номер $01$.

\item Совершенное множество --- замкнутое множество, не имеющее изолированных точек (точек, которые принадлежат внутренности дополнения).
%Покажите, что канторово множество совершенно. Покажите, что гомеоморфизм обязан сохранять совершенство пространства.
Используя это свойство, покажите, что топологическое пространство с <<коробочной топологией>> на $\{0,1\}^{\aleph_0}$ не гомеоморфно канторовому множеству.
На $\{0,1\}$ топология дискретна.

\end{enumerate}

\end{enumerate}

\section*{Домашнее задание №11: Объектно-ориентированное программирование и ТТ}
\begin{enumerate}
\item Покажите, что $\text{Pair }\tau_1\ \tau_2 := \forall \alpha<:\top.(\tau_1 \rightarrow\tau_2\rightarrow\alpha)\rightarrow\alpha$ обладает ковариантностью по обоим аргументам.
\item Какова вариантность алгебраического типа? Докажите. Какова вариантность неизменяемого списка? А изменяемого (<<мутабельного>>)?
\item Определим тип $\bot$: такой тип, что для всех $\alpha$ выполнено $\bot <: \alpha$. Покажите, что тип $\bot$ необитаем.
\item Определите в системе $F_{<:}$ квантор существования (аналогично тому, как это делается в $F$) и докажите правила, аналогичные правилам для $\forall$.
\item Постройте конструкцию в языке Java, использующую ядерное правило типизации $\forall$.
\item Закончите пример кода на Окамле, который был намечен в лекции: создайте объект shape, который будет иметь 
методы вычисления площади и печати названия, создайте пять разных фигур, которые будут иметь разную форму -- и которые
можно будет привести к типу shape. После этого напишите функции, вычисляющие общую площадь фигур из списка и печатающие
список фигур с площадями и названиями.
\item Определите на Окамле дерево разбора для арифметических выражений (объектно-ориентированное дерево с иерархией наследования,
константы и базовые операции $(+)$,$(-)$,$(\cdot)$,$(/)$), 
определите на нём шаблон <<визитор>> --- и с его помощью реализуйте две функции: подсчёт результат выражения и каноническая 
печать текста выражения.
\end{enumerate}

\section*{Домашнее задание №12: Дополнительные задачи на Arend}
\begin{enumerate}
\item Простые арифметические задачи:
\begin{enumerate}
\item Докажите, что $\forall x\in\mathbb{N}_0.(x+2)^2 = x^2 + 4x + 4$
\item Докажите, что $\forall x\in\mathbb{N}_0.(2x+1)^2 = 4x^2 + 4x + 1$
\item Докажите, что $\forall x\in\mathbb{N}_0.x^2 + 1 = 1 \rightarrow x = 0$
\item Докажите, что $\forall x\in\mathbb{N}_0.x < 10 \& x > 6 \rightarrow x = 7 \vee x=8 \vee x=9$
\item Докажите, что $\forall x\in\mathbb{N}_0.x < 3 \rightarrow x^2 < 9$
\item Докажите, что $\forall x\in\mathbb{N}_0.x^2 + x^2 + x^2 + x^2 = (2x)^2$
\item Докажите, что $\forall x\in\mathbb{N}_0.x \ge 1 \rightarrow x < 2 \rightarrow x = 1$
\item Докажите, что $\forall x\in\mathbb{N}_0.x = 0 \vee x = 1 \vee x = 2 \vee x=3 \vee x=4 \rightarrow x < 5$
\end{enumerate}

\item Бинарные отношения и арифметика. Пусть бинарные отношения представляются как зависимые типы (\verb!P: A -> A -> Type!,
где \verb!A! --- тип-носитель для отношения).
Тогда определите некоторое отношение и докажите его свойство:
\begin{enumerate}
\item Определите отношение <<меньше или равно>> на $\mathbb{N}_0$, покажите его ромбовидное свойство;
\item Определите отношение <<$a$ --- двоичное дополнение $b$>> на $0..n-1$, покажите его арефлексивность (нет $x$, что $a$ ---
двоичное дополнение $b$);
\item Определите отношение <<$a$ --- квадрат $b$>> на $\mathbb{N}_0$, покажите отсутствие свойства транзитивности;
\item Определите отношение <<$a$ не делится на $b$>> на $\mathbb{N}_0$, покажите отсутствие свойства симметричности;
\item Определите отношение <<$a$ имеет более длинную десятичную запись, чем $b$>> на $\mathbb{N}_0$, покажите отсутствие свойства рефлексивности;
\item Определите отношение <<двоичное представление $a$ --- сдвинутое влево и дополненное нулями справа двоичное представление $b$>> 
на $\mathbb{N}_0$, покажите его рефлексивность.
\end{enumerate}

\item Докажите, что линейное уравнение $ax = b$ при $a,b \in \mathbb{N}, a \ne 0$ имеет не более одного корня.

\item Определите тип <<множество элементов из сета $S$>> вместе с операцией $\in$ и отображением $P(Set\ S)$, задающим множество всех подмножеств. 
Покажите выполненость следующих аналогов конструктивных аксиом $ZF$:
\begin{enumerate}
\item Покажите, что для $a,b \in Set\ S$ выполнено $a = b$ тогда и только тогда, когда они состоят из одинаковых элементов ($\forall x:S.x \in a \rightarrow x \in b$).
\item Аксиома пары: если $p,q \in Set\ S$, то $\{p,q\} \in P(Set S)$.
\item Аксиома объединения: если $\forall i.p(i) \in Set\ S$, то $\cup_i p(i) \in Set\ S$.
\item Аксиома выделения: если $s \in Set\ S$ и $p : S \rightarrow \text{Prop}$, то $\{ x\in s | p(x)\} \in Set S$.
\item Выполнена ли аксиома выбора для таких множеств?
\item Пусть $p,q: Set\ S \rightarrow \text{Prop}$. Всегда ли $p(x) = p(y)$ эквивалентно $x=y$? Чего не хватает в определении, если не всегда?
\item Пусть $p: S \rightarrow \text{Prop}$ и пусть $A,B \in Set\ S$. Определите конструкцию $\bigvee_{x \in X^{Set\ S}}$ и 
покажите, что $\bigvee_{x \in A} p(x) \vee \bigvee_{x \in B} p(x)$ влечёт $\bigvee_{x\in A\cup B} p(x)$. 
\item Пусть $p: S \rightarrow \text{Prop}$ и пусть $A,B \in Set\ S$. Покажите, что если $\bigvee_{x\in A\cup B} p(x)$, то $\bigvee_{x \in A} p(x) \vee \bigvee_{x \in B} p(x)$.
Можете ли вы это показать как для \verb!||!, так и для \verb!Data.Or!?
\end{enumerate}

\item Определите тип <<неориентированный граф из $n$ вершин>>. Покажите следующие утверждения:
\begin{enumerate}
\item В полном графе без петель $\frac{n\cdot(n-1)}{2}$ дуг.
\item В любом связном графе не меньше $n-1$ дуги.
\item В дереве (связный граф без циклов) $n-1$ дуга.
\item В дереве каждые две вершины соединены единственным путём.
\item Дерево --- максимальный ациклический граф.
\end{enumerate}
\end{enumerate}

\section*{Дополнительные задания на баллы}
\begin{enumerate}
\item (15 баллов) Перенесите доказательство парадокса Жирара с языка Coq на Аренд:
\url{https://coq.inria.fr/doc/master/stdlib/Coq.Logic.Hurkens.html}

\item (15 баллов) 
Формализуйте и докажите малую теорему Ферма: если $a$ ($a \in \mathbb{N}$) не делится на простое число $p$,
то $$a^{p-1} \equiv 1 \mod p$$ 

\item (8 баллов) Для любых двух простых чисел, больших двойки, их сумма не является простым числом.

\item (15 баллов) Покажите, что для любых натуральных чисел $a_1,\dots,a_n$ найдутся такие константы $b,c$, что $\beta(b,c,i) = a_i$.

\item Определите тип <<перестановка>> из $n$ элементов (было ранее в дз). Каждая задача по 8 баллов.
\begin{enumerate}
\item Покажите, что перестановок длины $n$ ровно $n!$ штук.
\item Покажите, что перестановки образуют группу (стандартная библиотека Аренда).
%\item Покажите, что в перестановке есть элементы порядка $n$, но нет элементов большего порядка.
\end{enumerate}

\item (15 баллов) Определите свойство графа <<быть планарным>> и покажите формулу Эйлера для него.

\item (15 баллов) Покажите, что алгоритм Дейкстры на ориентированном графе действительно возвращает кратчайший путь. Рёбра в графе для простоты
пусть имеют вес 0 (отсутствует) или 1 (в наличии). Действия алгоритма Дейкстры, возможно, разумнее всего записывать в каком-то специально
построенном типе для записи шагов алгоритма (сами структуры данных каждого шага плюс утверждения про эти шаги).

\end{enumerate}

\end{document}
