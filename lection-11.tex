\documentclass[aspectratio=169,dvipsnames,usenames]{beamer}
\usepackage{graphicx}
\usepackage{stmaryrd}
\usepackage[T1,T2A]{fontenc}
\usepackage[utf8]{inputenc}
\usepackage[english,russian]{babel}
\usepackage{amsmath}
\usepackage{amsfonts}
\usepackage{amssymb}
\usepackage{makeidx}
\usepackage{verbatim}
\usepackage{amsthm}
%\usepackage{bnf}
\usepackage{tikz}
\usepackage{enumerate}
\usepackage{mathtext}
\usepackage{mathtools}
\usepackage{mathabx}
%\usepackage[left=2cm,right=2cm,top=2cm,bottom=2cm,bindingoffset=0cm]{geometry}
\usepackage{proof}
%\usepackage{paracol}
%\usepackage{enumitem}
\usepackage{xcolor}
\usepackage{colortbl}
%\usepackage{minted}
%\usepackage{hyperref}
\setbeamertemplate{navigation symbols}{}
\usetikzlibrary{graphs}
\usetikzlibrary{graphs.standard}
\usetikzlibrary{automata,positioning}
\usepackage{float}

\begin{document}

%\theoremstyle{dfn}
\newtheorem{dfn}{Определение}[section]
\newtheorem{nte}{Замечание}[section]

\newtheorem{axiom}{Аксиома}[section]
\newtheorem{thm}{Теорема}[section]
\newtheorem{lmm}[theorem]{Лемма}
\newtheorem{statement}{Утверждение}[section]
\newtheorem{oun_paragraph}{Пункт}[section]
\newtheorem{cons}{Следствие}[section]
\newtheorem*{exm}{Пример}

\newcommand{\comb}[1]{\operatorname{\mathcal{#1}}}
\newcommand{\func}[1]{\operatorname{#1}}
\newcommand{\reduction}[1]{{\color{OrangeRed}#1}}
\newcommand{\set}[1]{\left\{#1\right\}}

\def\from#1{\par \parbox{0.7\textwidth}{\par \hfill\raggedleft \it #1}} 

\begin{frame}{}
\begin{center}
{\LARGE Линейные и уникальные типы}
\end{center}
\end{frame}

\begin{frame}{Контекст требует формализации}

Напомним правила вывода:
	$$\infer{\Gamma, \theta \vdash \theta}{}
		\quad\infer{\Gamma \vdash \varphi \rightarrow \psi}{\Gamma, \varphi \vdash \psi}\quad
		\infer{\Gamma \vdash \psi}{\Gamma \vdash \varphi \to \psi && \Gamma \vdash \varphi}$$
	

Что такое контекст?
\begin{enumerate}
\item Это множество? Ведь $\alpha, \alpha\rightarrow\beta = \alpha\rightarrow\beta,\alpha$

	$$\infer{\alpha\rightarrow\beta, \alpha \vdash \beta}
                {\infer{\alpha, \alpha\rightarrow\beta \vdash \alpha\rightarrow\beta}{}\quad\quad\infer{ \alpha\rightarrow\beta,\alpha \vdash  \alpha}{}}$$
\item Разве это множество? Ведь $\alpha,\alpha \ne \alpha$.

	$$\infer{\vdash\alpha\rightarrow\alpha\rightarrow\alpha}{\infer{\alpha\vdash\alpha\rightarrow\alpha}{\infer{\alpha,\alpha\vdash\alpha}{}}}$$
\end{enumerate}

%Требуется формализация.
\end{frame}

\begin{frame}{Структурные правила}
Перестановка:
$$\infer{\Xi,\Sigma,\Delta,H\vdash\varphi}{\Xi,\Delta,\Sigma,H\vdash\varphi}$$
Сжатие:
$$\infer{\Xi,\delta\vdash\varphi}{\Xi,\delta,\delta\vdash\varphi}$$
Ослабление:
$$\infer{\Xi,\delta\vdash\varphi}{\Xi\vdash\varphi}$$

Сжатие и ослабление предполагают содержательные действия.

$$\infer{y:\beta\vdash\lambda x^\alpha.y : \alpha\rightarrow\beta}{\infer{y:\beta,x:\alpha\vdash y : \beta}{y:\beta\vdash y : \beta}}\quad\quad
  \infer{x:\alpha\vdash ( x,x ) : \alpha\times\alpha}{\infer{x:\alpha,x:\alpha\vdash ( x,x ) : \alpha \times \alpha}{x:\alpha\vdash x:\alpha\quad\quad x:\alpha\vdash x:\alpha}}$$
\end{frame}

\begin{frame}{Два варианта удаления конъюнкции}
Вариант 1 (классический):
$$\infer{\Gamma\vdash\alpha}{\Gamma\vdash\alpha\times\beta}\quad\infer{\Gamma\vdash\beta}{\Gamma\vdash\alpha\times\beta}$$

Вариант 2 (альтернативный):
$$\infer{\Gamma,\Delta\vdash\varphi}{\Gamma\vdash\alpha\times\beta\quad\Delta,\alpha,\beta\vdash\varphi}$$

Варианты эквивалентны, но только при наличии структурных правил:
$$\infer{\Gamma\vdash \alpha \times \beta}{\infer{\Gamma,\Gamma\vdash\alpha \times \beta}{\Gamma\vdash\alpha\quad\Gamma\vdash\beta}}
\quad\quad\quad
\infer{\Gamma\vdash\alpha}{\Gamma\vdash\alpha\times\beta\quad\infer{\alpha,\beta\vdash\alpha}{\infer{\alpha\vdash\alpha}{}}}
$$
\end{frame}

\begin{frame}{Изоморфизм Карри-Ховарда}
\begin{itemize}
\item Сжатие --- копирование значения
\item Ослабление --- удаление значения
\item Классическая конъюнкция:
$$\infer{\Gamma\vdash \text{fst}(P):\alpha}{\Gamma\vdash P:\alpha\times\beta}\quad\infer{\Gamma\vdash \text{snd}(P):\beta}{\Gamma\vdash P:\alpha\times\beta}$$

\item Альтернативная конъюнкция:
$$\infer{\Gamma,\Delta\vdash \text{case } P \text{ of }(x,y) \rightarrow E : \varphi}{\Gamma\vdash P:\alpha\times\beta\quad\Delta, x: \alpha, y: \beta\vdash E:\varphi}$$
\end{itemize}
\end{frame}

\begin{frame}{Линейная логика (вариант Филиппа Вадлера)}
Грамматика:
$$\varphi ::= X\ |\ \varphi\multimap\varphi\ |\ \varphi\otimes\varphi\ |\ \varphi\ \&\ \varphi\ |\ \varphi\oplus\varphi\ |\ \varphi\bindnasrepma\varphi\ |\ !\varphi$$

Два типа контекстов:
$\langle \alpha \rangle$ --- линейный,
$[\beta]$ --- интуиционистский

Структурные правила:

$$\infer{\Xi,\Delta,\Gamma,H\vdash\alpha}{\Xi,\Gamma,\Delta,H\vdash\alpha}\quad\quad
  \infer{\Gamma,[\alpha]\vdash\beta}{\Gamma,[\alpha],[\alpha]\vdash\beta}\quad\quad
  \infer{\Gamma,[\alpha]\vdash\beta}{\Gamma\vdash\beta}$$

Аксиомы:

$$\infer{[\alpha]\vdash\alpha}{}\quad\quad\infer{\langle\alpha\rangle\vdash\alpha}{}$$


\end{frame}

\begin{frame}{Правила для связок}
Правила для <<конечно>> (возможно тиражировать построение $\alpha$):

$$\infer{[\Gamma]\vdash !\alpha}{[\Gamma]\vdash\alpha}\quad\quad\infer{\Gamma,\Delta\vdash\beta}{\Gamma\vdash!\alpha\quad\quad\Delta,[\alpha]\vdash\beta}$$

Линейная импликация:
$$\infer{\Gamma\vdash\alpha\multimap\beta}{\Gamma,\langle\alpha\rangle\vdash\beta}\quad\quad\infer{\Gamma,\Delta\vdash\beta}{\Gamma\vdash\alpha\multimap\beta\quad\quad\Delta\vdash\alpha}$$
\end{frame}

\begin{frame}{Правила для связок: конъюнкция и дизъюнкция}
Мультипликативная конъюнкция (возможно построить и $\alpha$ и $\beta$ одновременно):
$$\infer{\Gamma,\Delta\vdash\alpha\otimes\beta}{\Gamma\vdash\alpha\quad\quad\Delta\vdash\beta}\quad\quad\infer{\Gamma,\Delta\vdash\varphi}{\Gamma\vdash\alpha\otimes\beta\quad\quad\Delta,\langle\alpha\rangle,\langle\beta\rangle\vdash\varphi}$$

Аддитивная конъюнкция (возможно построить $\alpha$ и возможно построить $\beta$, что-то одно по нашему выбору):
$$\infer{\Gamma\vdash\alpha\&\beta}{\Gamma\vdash\alpha\quad\quad\Gamma\vdash\beta}\quad\quad\infer{\Gamma\vdash\alpha}{\Gamma\vdash\alpha\&\beta}\quad\quad\infer{\Gamma\vdash\beta}{\Gamma\vdash\alpha\&\beta}$$

Аддитивная дизъюнкция (возможно построить $\alpha$ или возможно построить $\beta$, что-то одно по их выбору):
$$\infer{\Gamma\vdash\alpha\oplus\beta}{\Gamma\vdash\alpha}\quad\quad\infer{\Gamma\vdash\alpha\oplus\beta}{\Gamma\vdash\beta}\quad\quad
  \infer{\Gamma,\Delta\vdash\varphi}{\Gamma\vdash\alpha\oplus\beta\quad\quad\Delta,\langle\alpha\rangle\vdash\varphi\quad\quad\Delta,\langle\beta\rangle\vdash\varphi}$$

\end{frame}

\begin{frame}{Пример: интуиционистская импликация}
Введём обозначение: $$\alpha\rightarrow\beta := !\alpha\multimap\beta$$

Заметим, что такая импликация ведёт себя как интуиционистская:
$$\infer{\Gamma\vdash !\alpha\multimap\beta}{\infer{\Gamma,\langle!\alpha\rangle\vdash\beta}{\langle!\infer{\alpha\rangle\vdash!\alpha}{}\quad\quad\Gamma,[\alpha]\vdash\beta}}$$

$$\infer{\Gamma,[\Delta]\vdash\beta}{\Gamma\vdash!\alpha\multimap\beta\quad\quad\infer{[\Delta]\vdash!\alpha}{[\Delta]\vdash\alpha}}$$
\end{frame}

\begin{frame}{Вложение остальных интуиционистских связок в линейные}
Например, можно так:
\begin{center}\begin{tabular}{l}
$\alpha\rightarrow\beta := !\alpha\multimap\beta$\\
$\alpha\times\beta := \alpha\&\beta$\\
$\alpha+\beta := !\alpha\oplus!\beta$
\end{tabular}\end{center}

\end{frame}

\begin{frame}{Комбинаторный базис BCKW}

\begin{itemize}
\item $B := \lambda x.\lambda y.\lambda z.x\ (y\ z)$ (комбинатор $Z$, Zusammensetzung)
\item $C := \lambda x.\lambda y.\lambda z.x\ z\ y$ (комбинатор $T$, verTauschnung)
\item $K := \lambda x.\lambda y.x$ (Konstanz)
\item $W := \lambda x.\lambda y.x\ y\ y$
\end{itemize}
$BC$ --- линейная логика (значение нельзя ни удалить, ни скопировать)

$BCK$ --- аффинная логика (значение можно удалить, но не скопировать)

$BCKW$ --- интуиционистская логика
\vspace{1cm}

Почему? Например, $W:(\alpha\rightarrow\alpha\rightarrow\beta)\rightarrow(\alpha\rightarrow\beta)$.

\end{frame}

\begin{frame}{}
\begin{center}
{\LARGE Реализация\\Уникальные типы}
\end{center}
\end{frame}

\begin{frame}{Язык}
Роды:
$$\kappa ::= \mathcal{T}\ |\ \mathcal{U}\ |\ \star\ |\ \kappa\rightarrow\kappa$$

Типы:
\begin{center}\begin{tabular}{ll}
$\texttt{Int}$ & $: \mathcal{T}$\\
$\rightarrow$ & $: \star\rightarrow\star\rightarrow\mathcal{T}$\\
$\circ,\times$ & $: \mathcal{U}$\\
$\vee,\wedge$ & $: \mathcal{U}\rightarrow\mathcal{U}\rightarrow\mathcal{U}$\\
$\neg$ & $: \mathcal{U}\rightarrow\mathcal{U}$\\
$\texttt{Attr}$ & $: \mathcal{T}\rightarrow\mathcal{U}\rightarrow\star$
\end{tabular}\end{center}

Обозначения:
\begin{center}\begin{tabular}{l}
$t^u := \texttt{Attr}\ t\ u$\\
$a \rightarrow^u b := \texttt{Attr}\ (a\rightarrow b)\ u$
\end{tabular}\end{center}

Значения:
$$E ::= x^\odot\ |\ x^\otimes\ |\ \lambda x.E\ |\ E\ E$$
\end{frame}

\begin{frame}{Типизация}
Правила типизации имеют такой вид:
$$\Gamma\vdash E:\tau|_{fv}$$

Здесь $\Gamma$ сопоставляет переменным типы рода $\star$.
$fv$ сопоставляет переменным типы рода $\mathcal{U}$.

Правила вывода:
$$\infer{\Gamma,x:t^u\vdash x^\odot:t^u|_{x:u}}{}\quad\quad\infer{\Gamma,x:t^\times\vdash x^\otimes:t^\times|_{x:\times}}{}$$
$$\infer[fv' := fv\setminus\{x:\tau\}]{\Gamma\vdash\lambda x.E:a \rightarrow^{\bigvee fv'} b|_{fv'}}{\Gamma,x:a\vdash E:b|_{fv}}\quad\quad
  \infer{\Gamma\vdash E_1\ E_2:b|_{fv_1\cup fv_2}}{\Gamma\vdash E_1:a\rightarrow^u b|_{fv_1}\quad\quad\Gamma\vdash E_2:a|_{fv2}}$$
\end{frame}

\begin{frame}[fragile]{Смысл булевский выражений}
Будем считать уникальность истиной, а неуникальность --- ложью.
И рассмотрим, например, функцию \verb!fst!:

$$\texttt{fst} : (t^u, s^v)^{w\vee u} \rightarrow t^u$$

Это то же самое, что и

$$\texttt{fst} : (t^u, s^v)^w \rightarrow t^u, w \le u$$

Однако, подобные выражения могут быть разрешены с помощью \emph{булевской унификации}.
\end{frame}

\begin{frame}[fragile]{Где применяются линейные/уникальные типы}
Ручное распределение памяти:
\begin{verbatim}
void compute() {
    char* x = new char[1024];
    char* y = x;
    y[10] = 'a';
    delete [] x;      // delete y; нельзя! будут ошибки.
}
\end{verbatim}

Автоматическое распределение памяти (сборка мусора, подсчёт ссылок):
\begin{verbatim}
public void compute() {          fn compute() {
    int[] x = new int[1024];         let x = Arc::new("abcde");
    int[] y = x;                     let y = x.clone();
    x[10] = 15;                  }
}
\end{verbatim}

А давайте посчитаем количество ссылок при компиляции. Линейный тип всегда существует в единственном экземпляре.
\end{frame}

\end{document}
