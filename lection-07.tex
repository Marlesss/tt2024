\documentclass[aspectratio=169,dvipsnames,usenames]{beamer}
\usepackage{graphicx}
\usepackage{stmaryrd}
\usepackage[T1,T2A]{fontenc}
\usepackage[utf8]{inputenc}
\usepackage[english,russian]{babel}
\usepackage{amsmath}
\usepackage{amsfonts}
\usepackage{amssymb}
\usepackage{makeidx}
\usepackage{verbatim}
\usepackage{amsthm}
%\usepackage{bnf}
\usepackage{tikz}
\usepackage{enumerate}
\usepackage{mathtext}
\usepackage{mathtools}
\usepackage{mathabx}
%\usepackage[left=2cm,right=2cm,top=2cm,bottom=2cm,bindingoffset=0cm]{geometry}
\usepackage{proof}
%\usepackage{paracol}
%\usepackage{enumitem}
\usepackage{xcolor}
\usepackage{colortbl}
%\usepackage{minted}
%\usepackage{hyperref}
\setbeamertemplate{navigation symbols}{}
\usetikzlibrary{graphs}
\usetikzlibrary{graphs.standard}
\usetikzlibrary{automata,positioning}
\usepackage{float}

\begin{document}

%\theoremstyle{dfn}
\newtheorem{dfn}{Определение}[section]
\newtheorem{nte}{Замечание}[section]

\newtheorem{axiom}{Аксиома}[section]
\newtheorem{thm}{Теорема}[section]
\newtheorem{lmm}[theorem]{Лемма}
\newtheorem{statement}{Утверждение}[section]
\newtheorem{oun_paragraph}{Пункт}[section]
\newtheorem{cons}{Следствие}[section]
\newtheorem*{exm}{Пример}

\newcommand{\comb}[1]{\operatorname{\mathcal{#1}}}
\newcommand{\func}[1]{\operatorname{#1}}
\newcommand{\reduction}[1]{{\color{OrangeRed}#1}}
\newcommand{\set}[1]{\left\{#1\right\}}

\def\from#1{\par \parbox{0.7\textwidth}{\par \hfill\raggedleft \it #1}} 

\begin{frame}{}
\begin{center}
{\LARGE Иерархии универсумов}
\end{center}
\end{frame}

\begin{frame}[fragile]{Иерархия универсумов}
\begin{itemize}

\item Обобщённая типовая система:

\vspace{0.3cm}\begin{tabular}{lll}
Название сорта & Примеры сортов & Примеры конструкций, имеющих сорт\\\hline
Тип & $\alpha$, $\alpha\rightarrow\beta$, $\star\rightarrow\alpha$ & $3:\texttt{int}, \texttt{id}: \forall\alpha.\alpha\rightarrow\alpha$\\
Род (kind) & $\star$, $\star\rightarrow\star$, $\alpha\rightarrow\star$ & $\texttt{list}: \star\rightarrow\star$\\
Сорт & $\openbox$ & $\star\rightarrow\star : \openbox$
\end{tabular}

Также, сорт функции классифицируется по сорту её результата.

\item В Аренде сорт (тип) характеризуется двумя параметрами:

\begin{center}\verb!\h-Type m!\end{center}что также записывается как\begin{center}\verb!\Type m h!\end{center}

Здесь $h$ --- гомотопический уровень, $m$ --- предикативный уровень.
\end{itemize}

\end{frame}

\begin{frame}[fragile]{Prop и Set в Coq}
\begin{itemize}
\item Вселенная Prop --- типы чистых доказательств
\item Вселенная Set --- типы значений
\item Type --- совокупность Prop и Set
\item Ожидаем, что Prop сами исчезнут из кода конструктивных доказательств при оптимизации.

\item Однако, разделение на вселенные произвольное. Скажем, есть несколько доказательств того, что 15 --- составное:

$((3,5),3 \cdot 5 = 15) : \exists (x,y).x \cdot y = 15$
и
$((5,3),5 \cdot 3 = 15) : \exists (x,y).x \cdot y = 15$

Не всегда доказательство единственно.

\item Тип-сумма есть в двух вариантах:

\begin{center}\begin{tabular}{ll}
Prop & Set \\\hline
Дизъюнкция: \verb!a \/ b! & Тип-сумма: \verb!{a} + {b}!
\end{tabular}\end{center}
\end{itemize}

\begin{comment}
\small\color[HTML]{025002}
\begin{verbatim}
Inductive diveucl a b : Set :=
  divex : forall q r, b > r -> a = q * b + r -> diveucl a b.

Lemma eucl_dev : forall n, n > 0 -> forall m:nat, diveucl m n.

Lemma quotient :
  forall n,
    n > 0 ->
    forall m:nat, {q : nat | exists r : nat, m = q * n + r /\ n > r}.
\end{verbatim}
\end{comment}
\end{frame}

\begin{frame}[fragile]{Prop и Set в алгоритме Эвклида}
\begin{itemize}
\item Равенство --- это Prop:
\small\color[HTML]{025002}\begin{verbatim}
Inductive eq (A:Type) (x:A) : A -> Prop := eq_refl : x = x :>A
\end{verbatim}
\normalsize\color{black}
\item А типы <<a делится с остатком на b>>, <<$n \le m$ или $n > m$>> --- это Set:

\small\color[HTML]{025002}\begin{verbatim}
Inductive diveucl a b : Set :=
  divex : forall q r, b > r -> a = q * b + r -> diveucl a b.
Definition le_gt_dec n m : {n <= m} + {n > m}. ... Defined.
\end{verbatim}

\normalsize\color{black}
\item Конструктивное доказательство алгоритма Эвклида --- это смесь Prop и Set:
\small\color[HTML]{025002}
\begin{verbatim}
Lemma eucl_dev : forall n, n > 0 -> forall m:nat, diveucl m n.
Proof.
  intros b H a; pattern a in |- *; apply gt_wf_rec; intros n H0.
  elim (le_gt_dec b n).
  intro lebn. elim (H0 (n - b)); auto with arith.
  intros q r g e.
  apply divex with (S q) r; simpl in |- *; auto with arith.
  elim plus_assoc. elim e; auto with arith.
  intros gtbn.
  apply divex with 0 n; simpl in |- *; auto with arith.
Qed.
\end{verbatim}
\end{itemize}
\end{frame}

\begin{frame}[fragile]{Извлечение кода из доказательства}
Напомним определения:
\small\color[HTML]{025002}\begin{verbatim}
Inductive diveucl a b : Set :=
  divex : forall q r, b > r -> a = q * b + r -> diveucl a b.

Lemma eucl_dev : forall n, n > 0 -> forall m:nat, diveucl m n.
\end{verbatim}
\normalsize\color{black}

Извлечённый код содержит только относящиеся к Set термы:

\small\color[HTML]{025002}
\begin{verbatim}
type diveucl = Divex of nat * nat

let rec eucl_dev n m =
  let s = le_gt_dec n m in
  (match s with
   | Left ->
     let d = let y = sub m n in eucl_dev n y in
     let Divex (q, r) = d in Divex ((S q), r)
   | Right -> Divex (O, m))
\end{verbatim}
\end{frame}

\begin{frame}{Гомотопическая иерархия универсумов}
HoTT предлагает естественный смысл универсумам.

\begin{dfn}
Тип $\tau$ --- это $n$-тип, если тип равенств $\tau = \tau$ есть $(n-1)$-тип.
Универсум $Prop$ содержит $(-1)$-типы, все такие типы имеют ровно одно значение.
\end{dfn}

\begin{tabular}{lll}
Название & Уровень\\\hline
Prop & -1 & Значения единственны\\
Set & 0 & Путь между значениями единственен
\end{tabular}

\end{frame}

\begin{frame}{Гомотопическая иерархия в Аренде}
\begin{itemize}
\item Компилятор пытается угадать уровень.
\item Компилятору можно подсказать.
\item Уровень типа можно принудительно изменить.
\end{itemize}
\end{frame}

\begin{frame}[fragile]{Пример подсказки: разрешимость}
Разрешимое высказывание: такое, для которого 
существует алгоритм, возвращающий yes --- если высказывание
доказуемо, и no --- если доказуемо отрицание.
\small\color[HTML]{025002}
\begin{verbatim}
\data Dec (P : \Prop) | yes P | no (P -> Empty)
\end{verbatim}
\normalsize\color{black}

Выглядит как индуктивный тип с двумя вариантами (\verb!Dec P : \Set!). Однако, если
 $\vdash x : \texttt{yes}\ P$ и  $\vdash y : \texttt{no}\ (P \rightarrow \bot)$, то $\vdash (\dots) : \bot$.
Потому:

\small\color[HTML]{025002}
\begin{verbatim}
\data Dec (P : \Prop) | yes P | no (P -> Empty)
  \where
    \use \level isProp {P : \Prop} (d1 d2 : Dec P) : d1 = d2
      | yes x1, yes x2 => path (\lam i => yes (Path.inProp x1 x2 @ i))
      | yes x1, no e2 => \case e2 x1 \with {}
      | no e1, yes x2 => \case e1 x2 \with {}
      | no e1, no e2 => path (\lam i => no 
           (\lam x => (\case e1 x \return e1 x = e2 x \with {}) @ i)
        )
\end{verbatim}
\normalsize\color{black}
\end{frame}

\begin{frame}[fragile]{Пропозициональное усечение}
При необходимости тип можно обернуть в структуру с нужным гомотопическим уровнем.

\small\color[HTML]{025002}
\begin{verbatim}
\data TruncP (A : \Type)
  | inP A
  | truncP (a a' : TruncP A) : a = a'
  \where {
    \use \level levelProp {A : \Type} (a a' : TruncP A) : a = a' 
       => path (truncP a a')
  }
\end{verbatim}
\normalsize\color{black}

Или явно её задать с нужным уровнем (и нужным равенством):

\small\color[HTML]{025002}
\begin{verbatim}
\truncated \data Quotient {A : \Type} (R : A -> A -> \Type) : \Set
  | in~ A
  | ~-equiv (x y : A) (R x y) (i : I) \elim i {
    | left => in~ x
    | right => in~ y
  }
\end{verbatim}
\normalsize\color{black}

\end{frame}

\begin{frame}[fragile]{Логические конструкции и гомотопический уровень}
Конструкции, не повышающие гомотопический уровень: конъюнкция, импликация,
квантор всеобщности.

Это свойство можно сформулировать и доказать:

\small\color[HTML]{025002}
\begin{verbatim}
\func isProp (A : \Type) => \Pi (a a' : A) -> a = a'
\func and (a b : \Prop) : isProp (\Sigma a b) => \lam p q => {?}
\end{verbatim}
\normalsize\color{black}

\end{frame}

\begin{frame}[fragile]{Дизъюнкция в Аренде}
Повышает гомотопический уровень, не проп:
\small\color[HTML]{025002}
\begin{verbatim}
\data \fixr 2 Or (A B : \Type)
  | inl A
  | inr B
  \where {
    \func levelProp {A B : \Prop} (e : A -> B -> Empty) (x y : Or A B) 
        : x = y \elim x, y
      | inl a, inl a' => pmap inl (Path.inProp a a')
      | inl a, inr b => absurd (e a b)
      | inr b, inl a => absurd (e a b)
      | inr b, inr b' => pmap inr (Path.inProp b b')
  }
\end{verbatim}
\normalsize\color{black}

И есть усечённый до пропа вариант:
\small\color[HTML]{025002}
\begin{verbatim}
\truncated \data \infixr 2 || (A B : \Type) : \Prop
  | byLeft A
  | byRight B
\end{verbatim}
\normalsize\color{black}
\end{frame}

\begin{frame}[fragile]{Квантор существования в Аренде}
Квантор существования также увеличивает уровень. Не-проп вариант:

\small\color[HTML]{025002}
\begin{verbatim}
\func divides-set (a b : Nat) => \Sigma (p : Nat) (a = b Nat.* p)
\end{verbatim}
\normalsize\color{black}

Усечённый (проп) вариант:

\small\color[HTML]{025002}
\begin{verbatim}
\func divides' (a b : Nat) => TruncP (\Sigma (p : Nat) (a = b Nat.* p))

-- Встроенная в Аренд мета:
\func divides (a b : Nat) => Exists (p : Nat) (a = b Nat.* p)
\end{verbatim}
\normalsize\color{black}

Вместо \verb!Exists! можно писать $\exists$.

\end{frame}

\begin{frame}[fragile]{Каков тип типа?}

Напомним из обобщённой типовой системы:
$$\texttt{Nat} : \star\quad\quad\star : \openbox\quad\quad\openbox :\ ?$$

Такие сложные конструкции могут быть содержательны:
\small\color[HTML]{025002}
\begin{verbatim}
\func Church => \Pi (a : \Type) -> ((a -> a) -> a -> a)
\func zero : Church => \lam a (f : a -> a) (x : a) => x

\func inc : Church -> Church => \lam n 
    => \lam (a : \Type) (f : a -> a) (x : a) => n a f (f x)

\func add : Church -> Church -> Church => \lam m n => m Church inc n
\end{verbatim}
\normalsize\color{black}

\end{frame}

\begin{frame}{Парадокс Жирара}
Однако, попытки ввести тип всех типов легко могут привести к противоречию.

\begin{dfn}[система $U^-$]
$U^-$ --- это обобщённая типовая система с тремя сортами: $\{\star,\openbox,\triangle\}$,
двумя аксиомами: $$\vdash \star : \openbox\quad\quad\vdash \openbox : \triangle$$
и следующие частными правилами вывода: $$\{\langle \star,\star\rangle, \langle \openbox,\star\rangle,
        \langle \openbox,\openbox\rangle, \langle \triangle,\openbox\rangle\}$$
\end{dfn}

То есть, это $\lambda\omega$ с $\langle \triangle,\openbox\rangle$.

\begin{thm}[парадокс Жирара]
В типовой системе $U^-$ тип $\Pi p^\star.p$ обитаем.
\end{thm}
\end{frame}

\begin{frame}[fragile]{Предикативная иерархия}
\begin{dfn}
Атомарные типы имеют предикативный уровень 0.
Составной тип имеет тип с предикативным уровнем строго выше уровней его составных частей.

$$\varphi(\texttt{\textbackslash{}Type}\ m_1, \dots, \texttt{\textbackslash{}Type}\ m_n) : \texttt{\textbackslash{}Type}\ \max(m_1,\dots,m_n)+1$$
\end{dfn}

\begin{exm}
$(\star)$ можно понимать как аналог $\texttt{\textbackslash{}Type}\ 0$. Тогда
$(\star \rightarrow \star) \rightarrow \star : \texttt{\textbackslash{}Type}\ 1$.

Далее, $\texttt{\textbackslash{}Type}\ 1 : \texttt{\textbackslash{}Type}\ 2$ и т.п.
\end{exm}

\small\color[HTML]{025002}
\begin{verbatim}
\func star-star-star : (\Type 0 -> \Type 0) -> \Type 0 => {?}
\func id {t : \Type 1} (x : t) => x
\func x : (\Type 0 -> \Type 0) -> \Type 0 => 
          id {(\Type 0 -> \Type 0) -> \Type 0} star-star-star
\end{verbatim}
\end{frame}


\end{document}
